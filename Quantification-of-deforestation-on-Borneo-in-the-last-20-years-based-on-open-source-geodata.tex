% Options for packages loaded elsewhere
\PassOptionsToPackage{unicode}{hyperref}
\PassOptionsToPackage{hyphens}{url}
\PassOptionsToPackage{dvipsnames,svgnames,x11names}{xcolor}
%
\documentclass[
  letterpaper,
  DIV=11,
  numbers=noendperiod]{scrreprt}

\usepackage{amsmath,amssymb}
\usepackage{iftex}
\ifPDFTeX
  \usepackage[T1]{fontenc}
  \usepackage[utf8]{inputenc}
  \usepackage{textcomp} % provide euro and other symbols
\else % if luatex or xetex
  \usepackage{unicode-math}
  \defaultfontfeatures{Scale=MatchLowercase}
  \defaultfontfeatures[\rmfamily]{Ligatures=TeX,Scale=1}
\fi
\usepackage{lmodern}
\ifPDFTeX\else  
    % xetex/luatex font selection
\fi
% Use upquote if available, for straight quotes in verbatim environments
\IfFileExists{upquote.sty}{\usepackage{upquote}}{}
\IfFileExists{microtype.sty}{% use microtype if available
  \usepackage[]{microtype}
  \UseMicrotypeSet[protrusion]{basicmath} % disable protrusion for tt fonts
}{}
\makeatletter
\@ifundefined{KOMAClassName}{% if non-KOMA class
  \IfFileExists{parskip.sty}{%
    \usepackage{parskip}
  }{% else
    \setlength{\parindent}{0pt}
    \setlength{\parskip}{6pt plus 2pt minus 1pt}}
}{% if KOMA class
  \KOMAoptions{parskip=half}}
\makeatother
\usepackage{xcolor}
\setlength{\emergencystretch}{3em} % prevent overfull lines
\setcounter{secnumdepth}{5}
% Make \paragraph and \subparagraph free-standing
\ifx\paragraph\undefined\else
  \let\oldparagraph\paragraph
  \renewcommand{\paragraph}[1]{\oldparagraph{#1}\mbox{}}
\fi
\ifx\subparagraph\undefined\else
  \let\oldsubparagraph\subparagraph
  \renewcommand{\subparagraph}[1]{\oldsubparagraph{#1}\mbox{}}
\fi


\providecommand{\tightlist}{%
  \setlength{\itemsep}{0pt}\setlength{\parskip}{0pt}}\usepackage{longtable,booktabs,array}
\usepackage{calc} % for calculating minipage widths
% Correct order of tables after \paragraph or \subparagraph
\usepackage{etoolbox}
\makeatletter
\patchcmd\longtable{\par}{\if@noskipsec\mbox{}\fi\par}{}{}
\makeatother
% Allow footnotes in longtable head/foot
\IfFileExists{footnotehyper.sty}{\usepackage{footnotehyper}}{\usepackage{footnote}}
\makesavenoteenv{longtable}
\usepackage{graphicx}
\makeatletter
\def\maxwidth{\ifdim\Gin@nat@width>\linewidth\linewidth\else\Gin@nat@width\fi}
\def\maxheight{\ifdim\Gin@nat@height>\textheight\textheight\else\Gin@nat@height\fi}
\makeatother
% Scale images if necessary, so that they will not overflow the page
% margins by default, and it is still possible to overwrite the defaults
% using explicit options in \includegraphics[width, height, ...]{}
\setkeys{Gin}{width=\maxwidth,height=\maxheight,keepaspectratio}
% Set default figure placement to htbp
\makeatletter
\def\fps@figure{htbp}
\makeatother
\newlength{\cslhangindent}
\setlength{\cslhangindent}{1.5em}
\newlength{\csllabelwidth}
\setlength{\csllabelwidth}{3em}
\newlength{\cslentryspacingunit} % times entry-spacing
\setlength{\cslentryspacingunit}{\parskip}
\newenvironment{CSLReferences}[2] % #1 hanging-ident, #2 entry spacing
 {% don't indent paragraphs
  \setlength{\parindent}{0pt}
  % turn on hanging indent if param 1 is 1
  \ifodd #1
  \let\oldpar\par
  \def\par{\hangindent=\cslhangindent\oldpar}
  \fi
  % set entry spacing
  \setlength{\parskip}{#2\cslentryspacingunit}
 }%
 {}
\usepackage{calc}
\newcommand{\CSLBlock}[1]{#1\hfill\break}
\newcommand{\CSLLeftMargin}[1]{\parbox[t]{\csllabelwidth}{#1}}
\newcommand{\CSLRightInline}[1]{\parbox[t]{\linewidth - \csllabelwidth}{#1}\break}
\newcommand{\CSLIndent}[1]{\hspace{\cslhangindent}#1}

\KOMAoption{captions}{tableheading}
\usepackage{fontspec}
\setmainfont{Arial}
\setsansfont{Arial}
\usepackage{anyfontsize}
\fontsize{11}{16}\selectfont
\usepackage{booktabs}
\usepackage{graphicx}
\usepackage[nottoc]{tocbibind}
\usepackage{natbib}
\PassOptionsToPackage{table}{xcolor}
\usepackage{xcolor}
\renewcommand{\bibsection}{}
\usepackage[a4paper, top=2.5cm, bottom=3cm, left=2.5cm, right=2.5cm]{geometry}
\usepackage{tocloft}
\setlength{\cftbeforetoctitleskip}{0pt}
\renewcommand{\cftsecleader}{\cftdotfill{\cftdotsep}}
\usepackage{scrhack}
\usepackage{etoolbox}
\makeatletter
\patchcmd{\chapter}{\if@openright\cleardoublepage\else\clearpage\fi}{}{}{}
\makeatother
\usepackage{caption}
\captionsetup[figure]{font={footnotesize}, labelfont={bf}}
\captionsetup[table]{font={footnotesize}, labelfont={bf}}
\usepackage{chngcntr}
\counterwithout{figure}{chapter}
\counterwithout{table}{chapter}
\RedeclareSectionCommand[ beforeskip=12pt plus 1pt minus 1pt, afterskip=6pt plus 1pt minus 1pt, font=\fontsize{16pt}{20pt}\bfseries\sffamily, ]{chapter}
\RedeclareSectionCommand[ beforeskip=10pt plus 1pt minus 1pt, afterskip=5pt plus 1pt minus 1pt, font=\fontsize{14pt}{18pt}\bfseries\sffamily, ]{section}
\RedeclareSectionCommand[ beforeskip=8pt plus 1pt minus 1pt, afterskip=4pt plus 1pt minus 1pt, font=\fontsize{12pt}{16pt}\bfseries\sffamily, ]{subsection}
\RedeclareSectionCommand[ beforeskip=6pt plus 1pt minus 1pt, afterskip=3pt plus 1pt minus 1pt, font=\fontsize{10pt}{14pt}\bfseries\sffamily, ]{subsubsection}
\RedeclareSectionCommand[ beforeskip=4pt plus 1pt minus 1pt, afterskip=2pt plus 1pt minus 1pt, font=\fontsize{10pt}{12pt}\bfseries\sffamily, ]{paragraph}
\RedeclareSectionCommand[ beforeskip=2pt plus 1pt minus 1pt, afterskip=1pt plus 1pt minus 1pt, font=\fontsize{8pt}{10pt}\bfseries\sffamily, ]{subparagraph}
\makeatletter
\makeatother
\makeatletter
\@ifpackageloaded{bookmark}{}{\usepackage{bookmark}}
\makeatother
\makeatletter
\@ifpackageloaded{caption}{}{\usepackage{caption}}
\AtBeginDocument{%
\ifdefined\contentsname
  \renewcommand*\contentsname{Table of contents}
\else
  \newcommand\contentsname{Table of contents}
\fi
\ifdefined\listfigurename
  \renewcommand*\listfigurename{List of Figures}
\else
  \newcommand\listfigurename{List of Figures}
\fi
\ifdefined\listtablename
  \renewcommand*\listtablename{List of Tables}
\else
  \newcommand\listtablename{List of Tables}
\fi
\ifdefined\figurename
  \renewcommand*\figurename{Figure}
\else
  \newcommand\figurename{Figure}
\fi
\ifdefined\tablename
  \renewcommand*\tablename{Table}
\else
  \newcommand\tablename{Table}
\fi
}
\@ifpackageloaded{float}{}{\usepackage{float}}
\floatstyle{ruled}
\@ifundefined{c@chapter}{\newfloat{codelisting}{h}{lop}}{\newfloat{codelisting}{h}{lop}[chapter]}
\floatname{codelisting}{Listing}
\newcommand*\listoflistings{\listof{codelisting}{List of Listings}}
\makeatother
\makeatletter
\@ifpackageloaded{caption}{}{\usepackage{caption}}
\@ifpackageloaded{subcaption}{}{\usepackage{subcaption}}
\makeatother
\makeatletter
\@ifpackageloaded{tcolorbox}{}{\usepackage[skins,breakable]{tcolorbox}}
\makeatother
\makeatletter
\@ifundefined{shadecolor}{\definecolor{shadecolor}{rgb}{.97, .97, .97}}
\makeatother
\makeatletter
\makeatother
\makeatletter
\makeatother
\ifLuaTeX
  \usepackage{selnolig}  % disable illegal ligatures
\fi
\IfFileExists{bookmark.sty}{\usepackage{bookmark}}{\usepackage{hyperref}}
\IfFileExists{xurl.sty}{\usepackage{xurl}}{} % add URL line breaks if available
\urlstyle{same} % disable monospaced font for URLs
\hypersetup{
  pdfauthor={Robin Pfaff},
  colorlinks=true,
  linkcolor={black},
  filecolor={Maroon},
  citecolor={Blue},
  urlcolor={Blue},
  pdfcreator={LaTeX via pandoc}}

\author{Robin Pfaff}
\date{2023-10-29}

\begin{document}
\begin{titlepage}
    \centering
    {\fontsize{12}{10}\selectfont ZURICH UNIVERSITY OF APPLIED SCIENCES\par}
    {\fontsize{12}{10}\selectfont SCHOOL OF LIFE SCIENCES AND FACILITY MANAGEMENT\par}
    {\fontsize{12}{10}\selectfont INSTITUTE OF NATURAL RESOURCE SCIENCES\par}
    \vspace{6cm}
    {\fontsize{14}{16}\bfseries Quantification of deforestation on Borneo in the last 20 years based on open source geodata\par}
    {\fontsize{12}{14}\bfseries Bachelor Thesis\par}
    {\fontsize{12}{14}\selectfont HS23\par}
    \vspace{2cm}
    {\fontsize{12}{14}\bfseries by\par}
    {\fontsize{12}{14}\bfseries Robin Pfaff\par}
    {\fontsize{12}{14}\selectfont BSc Environmental engineering\par}
    \vspace{2cm}
    {\fontsize{12}{14}\selectfont Submission date: 29.10.2023\par}

    \vfill
    \begin{flushleft}
        1\textsuperscript{st} supervisor:\\
        Ochsner, Pascal\\
        2\textsuperscript{nd} supervisor:\\
        Ratnaweera, Nils\\
        ZHAW IUNR Research Group for Geoinformatics
    \end{flushleft}
\end{titlepage}

\clearpage
\pagenumbering{arabic}
\setcounter{page}{1}

\thispagestyle{empty}
\vspace*{\fill}

\noindent{\Large\textbf{Imprint}}

\vspace{0.5cm}

\noindent\textbf{Institute}\\
Institute of Natural Resource Sciences

\noindent\textbf{Form of citation}\\
APA 7th edition

\noindent\textbf{Keywords}\\
deforestation, spatiotemporal analysis, open source data, Boreno, GIS

\ifdefined\Shaded\renewenvironment{Shaded}{\begin{tcolorbox}[enhanced, breakable, borderline west={3pt}{0pt}{shadecolor}, interior hidden, frame hidden, sharp corners, boxrule=0pt]}{\end{tcolorbox}}\fi

\bookmarksetup{startatroot}

\hypertarget{abstract}{%
\chapter*{Abstract}\label{abstract}}

\markboth{Abstract}{Abstract}

\newpage
\tableofcontents

\bookmarksetup{startatroot}

\hypertarget{sec-introduction}{%
\chapter{Introduction}\label{sec-introduction}}

Borneo, the third largest island in the world, is home to extensive
tropical forests and peatlands. They have a vital ecological and
climatic role on a global scale and are of great socioeconomic value on
a national and regional level
(\protect\hyperlink{ref-harrisonTropicalForestPeatland2020}{Harrison et
al., 2020}). With an estimated 10,000 to 15,000 species of flowering
plants, 37 endemic bird and 44 endemic mammal species
(\protect\hyperlink{ref-mackinnonEcologyKalimantan1997}{MacKinnon et
al., 1997}), Borneo is part of the Sundaland biodiversity hotspot
(\protect\hyperlink{ref-myersBiodiversityHotspotsConservation2000}{Myers
et al., 2000}). However, this vast flora and fauna is threatened by land
use change, climate change and fire with the IUCN listing 415 species as
threatened
(\protect\hyperlink{ref-harrisonTropicalForestPeatland2020}{Harrison et
al., 2020}). Among them is the humans' closest relative, the critically
endangered Borneo Orangutan (\emph{Pongo pygmaeus},
\protect\hyperlink{ref-iucnPongoPygmaeusAncrenaz2016}{IUCN, 2016}). In
addition to biodiversity loss, these factors also have profound
consequences at regional and global levels. Among them are restrained
water quality and quantity regulation services and release of GHGs
(\protect\hyperlink{ref-foleySolutionsCultivatedPlanet2011}{Foley et
al., 2011}). Because of these far-reaching consequences, measures to
halt, protect, or even reverse tropical deforestation have received
increased attention. This, for example, through incorporating regulatory
standards, corporate voluntary sustainability commitments, protected
area networks, economic incentives, and demand-side interventions
(\protect\hyperlink{ref-austinWhatCausesDeforestation2019}{Austin et
al., 2019}).

Vegetable oil producing crops such as canola, soybean, sunflower, and
oil palm occupy \textasciitilde7.5\%~of the world's agricultural land
with a total annual production of 217 Mt (2020 - 2022
\protect\hyperlink{ref-oecdOECDFAOAgriculturalOutlook2023}{OECD, 2023}).
This demand is expected to increase to 310 Mt by mid-century
(\protect\hyperlink{ref-byerleeTropicalOilCrop2016}{Byerlee et al.,
2016}), due to the increasing world population and as a renewable
resource for biofuel
(\protect\hyperlink{ref-abdulmajidSustainablePalmOil2021}{Abdul Majid et
al., 2021}). With a share of approximately one-third, palm oil is
globally the most widely used vegetable oil
(\protect\hyperlink{ref-kamyabElaeisGuineensis2022}{Kamyab, 2022};
\protect\hyperlink{ref-oecdOECDFAOAgriculturalOutlook2023}{OECD, 2023}).
Borneo, which is split between the two largest palm oil producers in the
world - Malaysia and Indonesia produce 85\% of the world's palm oil - is
a cultivation hotspot due to its tropical climate and high rainfalls
(\protect\hyperlink{ref-kamyabElaeisGuineensis2022}{Kamyab, 2022}). In
addition to oil palm, other crops such as rice, groundnut, cassava,
coffee, cocoa, maize, rubber, coconut, and pulpwood are also cultivated
in Malaysia and Indonesia
(\protect\hyperlink{ref-faoFAOSTATDatabase2023}{FAO, 2023}). However,
their harvest areas are subject to much less expansion and therefore,
apart from coconut
(\protect\hyperlink{ref-meijaardCoconutOilConservation2020}{Meijaard et
al., 2020}), do not pose a major problem in landscape ecology in this
region (\protect\hyperlink{ref-potapovGlobalMapsCropland2021}{Potapov et
al., 2021}).

Another driver closely linked to forest loss is expansion of
infrastructure
(\protect\hyperlink{ref-meijerGlobalPatternsCurrent2018}{Meijer et al.,
2018};
\protect\hyperlink{ref-sloanHiddenChallengesConservation2019}{Sloan et
al., 2019}). Especially in Asia, large infrastructure expansions
proceeded in the last two decades
(\protect\hyperlink{ref-potapovGlobal20002020Land2022}{Potapov et al.,
2022}). Besides forest loss, new infrastructure is also linked to
increased hunting pressure and light pollution
(\protect\hyperlink{ref-kamyabElaeisGuineensis2022}{Kamyab, 2022};
\protect\hyperlink{ref-lewanzikArtificialLightPuts2014}{Lewanzik \&
Voigt, 2014}).

Although many recent global datasets provide the basis for analyzing the
key drivers of deforestation, the most recent study examining this for
Borneo was conducted a decade ago, without focus on other drivers
besides oil palm
(\protect\hyperlink{ref-gaveauFourDecadesForest2014}{Gaveau et al.,
2014}).

Over the past decade, open source GIS has steadily improved and gained
relevance in academia and industry
(\protect\hyperlink{ref-coetzeeOpenGeospatialSoftware2020}{Coetzee et
al., 2020};
\protect\hyperlink{ref-mobasheriHighlightingRecentTrends2020}{Mobasheri,
Mitasova, et al., 2020}). Combined with open-source geospatial data,
which became widely available since 2008
(\protect\hyperlink{ref-zhuBenefitsFreeOpen2019}{Zhu et al., 2019}), it
promises to have the potential to accelerate the solution of global and
interdisciplinary problems
(\protect\hyperlink{ref-coetzeeOpenGeospatialSoftware2020}{Coetzee et
al., 2020};
\protect\hyperlink{ref-mobasheriOpensourceGeospatialTools2020}{Mobasheri,
Pirotti, et al., 2020}). Moreover, machine learning algorithms allowed
classification into land use by recognition of repeating patterns
(\protect\hyperlink{ref-crowleyRemoteSensingRecent2020}{Crowley \&
Cardille, 2020}). These advances have been widely applied for mapping of
deforestation
(\protect\hyperlink{ref-curtisClassifyingDriversGlobal2018}{Curtis et
al., 2018};
\protect\hyperlink{ref-hansenHighResolutionGlobalMaps2013}{Hansen et
al., 2013}) and detection of oil plam
(\protect\hyperlink{ref-danyloMapExtentYear2021}{Danylo et al., 2021};
\protect\hyperlink{ref-descalsHighresolutionGlobalMap2021}{Descals et
al., 2021}).

Building upon these modern open-source datasets, this thesis
investigates deforestation and its drivers, within the spatial extent of
Borneo, and a temporal scope spanning the past two decades. More
precise, the following questions are addressed: (i) what is the yearly
extent of deforestation, (ii) which influence did the introduction of
the Roundtable for Sustainable Palm Oil (RSPO) have on deforestation
rates, (iii) how much existing cropland was converted into oil palm
plantations and (iv) how does proximity to infrastructure impact
deforestation rates? Annex ii provides a list of the in-depth questions
that help to answer these guiding questions.

Considering the global importance of this region, providing answers to
these questions is crucial to aid policy makers monitor global
initiatives for sustainable development, climate change mitigation, and
conservation of biodiversity and ecosystem functions at the regional
level of Borneo. These initiatives include the United Nations Framework
Convention on Climate Change, the Paris Agreement and COP26 Glasgow
Declaration, the Convention on Biological Diversity and the UN
Sustainable Development Goals (SDGs). Especially SDGs responsible
consumption and production (SDG 12), cilmate action (SDG 13) and life on
land (SDG 15), as well as to a smaller extent life under water (SDG 14)
and clean water and sanitation (SDG 6). Furthermore, NGOs operating on
Borneo (e.g.~Borneo Orangutan Survival Foundation, Bruno Manser Fonds,
etc.) can use these results for the selection of new projects as well as
for communication to raise public awareness on deforestation and its
drivers.

\bookmarksetup{startatroot}

\hypertarget{literature-review}{%
\chapter{Literature review}\label{literature-review}}

\hypertarget{sec-deforestation}{%
\section{Deforestation}\label{sec-deforestation}}

Forests are among the most important terrestrial ecosystems and are
fundamental to all life processes
(\protect\hyperlink{ref-zafirahSustainableEcosystemServices2017}{Zafirah
et al., 2017}). Tropical forests play a more influential role in the
climate cycle and various biodiversity processes in comparison to other
terrestrial biomes
(\protect\hyperlink{ref-zafirahSustainableEcosystemServices2017}{Zafirah
et al., 2017}). Their ecosystem services include regulation of the water
cycle (quality and quantity), carbon storage, natural pest control,
climate regulation, pollination and seed dispersal
(\protect\hyperlink{ref-nasiForestEcosystemServices2002}{Nasi et al.,
2002}). Undoubtedly, they possess the highest biological diversity, with
estimates that consider the global share to be more than half
(\protect\hyperlink{ref-jenkinsGlobalPatternsTerrestrial2013}{Jenkins et
al., 2013}). Additionally, tropical forests are also indispensable for
social and cultural identity, livelihoods and climate change adaptation
and mitigation
(\protect\hyperlink{ref-omettoContributionWorkingGroup2022}{Ometto et
al., 2022}).

However, tropical forests are threatened. Reports estimate that almost
half (\protect\hyperlink{ref-hansenHighResolutionGlobalMaps2013}{Hansen
et al., 2013}), or even nine-tenths
(\protect\hyperlink{ref-faoGlobalForestResources2020}{FAO, 2020}), of
global deforestation is happening in the tropics, of which rainforests
have the highest share
(\protect\hyperlink{ref-hansenHighResolutionGlobalMaps2013}{Hansen et
al., 2013}). In 2010 Gaveau et al.~indicated that three-quarters of
Borneo was covered with forest in the early 1970s, reducing to only
390,000 km\textsuperscript{2} (52.8\%) remaining forested of which
210,000 km\textsuperscript{2} are considered intact
(\protect\hyperlink{ref-gaveauFourDecadesForest2014}{2014}). Much of
this deforestation is due to conversion to large-scale agricultural
plantations (mainly soybeans, oil palm, corn, cotton, and livestock; see
Section~\ref{sec-oilpalm}), which represents one of the main causes of
the decline in species richness in flora and fauna
(\protect\hyperlink{ref-jaureguiberryDirectDriversRecent2022}{Jaureguiberry
et al., 2022};
\protect\hyperlink{ref-omettoContributionWorkingGroup2022}{Ometto et
al., 2022}). Forests also prevent nutrient runoff and erosion due to
their vegetation cover
(\protect\hyperlink{ref-sweeneyStreamsideForestBuffer2014}{Sweeney \&
Newbold, 2014}). Subsequently, deforestation can lead to enhanced
sedimentation and thus shallower river networks. This in return can
cause massive floods during monsoon season
(\protect\hyperlink{ref-zafirahSustainableEcosystemServices2017}{Zafirah
et al., 2017}).

The protection of the forest is also important because it serves as
CO\textsubscript{2} reservoir. These carbon stocks get released by
deforestation and account for 10-15\% of anthropogenic GHGs
(\protect\hyperlink{ref-houghtonEmissionsCarbonDeforestation2013}{Houghton,
2013}). Even if cultivated trees are subsequently grown, they cannot
retain the same amount of CO\textsubscript{2}
(\protect\hyperlink{ref-waringForestsDecarbonizationRoles2020}{Waring et
al., 2020}). In 2011, as a reaction to the deforestation rates,
Indonesia instituted a moratorium on new licenses for oil palm and
timber plantations as well as logging on primary forests and peat lands
(\protect\hyperlink{ref-buschReductionsEmissionsDeforestation2015}{Busch
et al., 2015}). Busch et al.~estimate that an additional 241 - 615
MtCO\textsubscript{2} equivalents could have been saved if the
moratorium had already been implemented in 2000
(\protect\hyperlink{ref-buschReductionsEmissionsDeforestation2015}{2015}).

\begin{itemize}
\tightlist
\item
  vielleicht noch mehr auf kreisläufe eingehen/zusammebringen /
  sequestrierung von CO2 Phosphor, Stickstoff.
\item
  evtl mehr von Houghton 2013
\end{itemize}

Another major problem is forest degradation. Matricardi et al.~state
that forest degradation in the tropics occurs over even larger areas
than it is deforested
(\protect\hyperlink{ref-matricardiLongtermForestDegradation2020}{2020}).
However, this is much more difficult to measure because it occurs in the
forest and leaves a closed canopy, making it challenging to use remote
sensing methods.

\hypertarget{sec-oilpalm}{%
\section{\texorpdfstring{Oil palm \emph{(Elaeis
guineensis)}}{Oil palm (Elaeis guineensis)}}\label{sec-oilpalm}}

Originating from West Africa, the oil palm \emph{(Elaeis guineensis)} is
the most efficient and important oil-producing crop worldwide, with a
lifespan of \textasciitilde25 years
(\protect\hyperlink{ref-kamyabElaeisGuineensis2022}{Kamyab, 2022}). It
has the highest yield in tropical climates with high rainfall
(\protect\hyperlink{ref-kamyabElaeisGuineensis2022}{Kamyab, 2022}).
Hence it is successfully cultivated in Malaysia and Indonesia, with
yields ranging from 10 to 35 tons of fresh fruit bunch (FFB) per hectare
(\protect\hyperlink{ref-kamyabElaeisGuineensis2022}{Kamyab, 2022}).
These two countries produce 85\% of the world's palm oil
(\protect\hyperlink{ref-kamyabElaeisGuineensis2022}{Kamyab, 2022};
\protect\hyperlink{ref-ritchiePalmOil2021}{Ritchie, 2021}).
Figure~\ref{fig-op_yield} shows that in less than 20 years globall
harvest area has almost quadrupled from 8 Mha in 1994 to 29 Mha in 2021
(\protect\hyperlink{ref-faoFAOSTATDatabase2023}{FAO, 2023}). Therefore,
it was extensively mapped using the characteristic backscatter response
of palm-like trees from remoteley sensed data
(\protect\hyperlink{ref-descalsHighresolutionGlobalMap2021}{Descals et
al., 2021}).

While the massive expansion of production capacities over the last
decades is associated with economic growth and rapid development, the
ecological impact of converting forest into giant oil palm monocultures
and the environmental pollution caused by large quantities of
by-products, mainly palm oil mill effluent (POME), during oil extraction
pose major problems
(\protect\hyperlink{ref-kamyabElaeisGuineensis2022}{Kamyab, 2022}). The
revenue from previous logging of old growth forest helps to subsidize
the initial plantation costs
(\protect\hyperlink{ref-fitzherbertHowWillOil2008}{Fitzherbert et al.,
2008}). However, the conversion is estimated to lose up to 45\% of
species diversity, density and biomass of invertebrate communities
(\protect\hyperlink{ref-barnesConsequencesTropicalLand2014}{Barnes et
al., 2014}). Moreover, the drainage of peatlands for land preparation
causes the release of large amounts of GHGs
(\protect\hyperlink{ref-sheilImpactsOpportunitiesOil2009}{Sheil, 2009}).
These expansions of harvested areas ae also closely linked to changing
market values
(\protect\hyperlink{ref-gaveauSlowingDeforestationIndonesia2022}{Gaveau
et al., 2022}).

Per ton of FFB, 600-700 kg of POME are produced
(\protect\hyperlink{ref-kamyabElaeisGuineensis2022}{Kamyab, 2022}). POME
is effluent from the mills and has a high content of chemical oxygen
demand (COD), biochemical oxygen demand (BOD) and heavy metals
(\protect\hyperlink{ref-hadiyantoPhytoremediationsPalmOil2013}{Hadiyanto
et al., 2013}). It also carries large amounts of nutrients such as
nitrogen or phosphorus
(\protect\hyperlink{ref-jeongPerformanceComparisonMesophilic2014}{Jeong
et al., 2014}). This renders it highly toxic to aquatic life forms
(\protect\hyperlink{ref-kamyabElaeisGuineensis2022}{Kamyab, 2022}).
Ultimately, this effluent, insufficiently treated in the majority of
cases, leads to health risks for the population living along waters with
oil palm mills upstream, eventually even affecting crops irrigated with
this water (\protect\hyperlink{ref-kamyabElaeisGuineensis2022}{Kamyab,
2022}).

Palm oil has been used mainly in the food (80\%) and to a lesser extent
in the non-food sector (20\%) including biodiesel
(\protect\hyperlink{ref-basironOILPALMITS2004}{Basiron \& Weng, 2004}).
Today, 68\% are used in foods, 27\% in industrial applications and 5\%
is used for bioenergy
(\protect\hyperlink{ref-ritchiePalmOil2021}{Ritchie, 2021}). 87\% of the
European Union's (EU) palm oil imports in 2017 were processed into
biodiesel
(\protect\hyperlink{ref-abdulmajidSustainablePalmOil2021}{Abdul Majid et
al., 2021}). This demand is expected to increase even more as the EU has
solid legislation around biofuel production
(\protect\hyperlink{ref-abdulmajidSustainablePalmOil2021}{Abdul Majid et
al., 2021}). But global population growth and the simultaneous increase
in demand for vegetable oil will also further increase demand in this
area.

Thus, the market urgently requires solutions to reduce its environmental
impact. One solution to meet the growing demand for palm oil without
expanding cropland is to increase oil extraction rates. Nevertheless,
despite research in this area, the OERs have stagnated at 19-21\% over
the past 40 years
(\protect\hyperlink{ref-changEconomicPerspectiveOil2003}{Chang et al.,
2003}; \protect\hyperlink{ref-chewImprovingSustainabilityPalm2021}{Chew
et al., 2021}). Another option to reduce pressure on land conversion is
genomic research to improve yields, with various approaches being taken
(\protect\hyperlink{ref-murphyOilPalm2020s2021}{Murphy et al., 2021}). A
private producer claims that his research has the potential to increase
yields by up to 20\%
(\protect\hyperlink{ref-simedarbyplantationSimeDarbyPlantation2020}{Sime
Darby Plantation, 2020}). However, after a remarkable 40\% increase in
yield from 1994 to 2011, it has since stagnated at 15 tons per hectare
(see Figure~\ref{fig-op_yield}) and such advances take time for
implementation due to the \textasciitilde25 year lifespan of
plantations.

\begin{figure}

{\centering \includegraphics{text/04_literature_review_files/op_yield.png}

}

\caption{\label{fig-op_yield}Global harvest area and production of oil
palm evolve accordingly, with no yield improvement since 2011
(\protect\hyperlink{ref-faoFAOSTATDatabase2023}{FAO, 2023}).}

\end{figure}

In summary, oil palm is an essential crop globally, known for its
efficiency and high yields in tropical regions. While it has contributed
significantly to economic growth, its expansion has come at the cost of
deforestation and pollution, posing challenges to biodiversity and
ecosystems. Nonetheless, markets will continue to thrive, requiring new
solutions to reduce land use change and innovations in processing.

\hypertarget{sustainable-palm-oil-labels}{%
\section{Sustainable Palm Oil
Labels}\label{sustainable-palm-oil-labels}}

Growing concern about the significant forest loss and ecosystem
degradation caused by deforestation has brought cultivation of palm oil
to public attention. As a result, the Roundtable on Sustainable Palm Oil
(RSPO) was established in 2004 to build a market for certified
sustainable palm oil by establishing environmental and social standards
for palm oil production
(\protect\hyperlink{ref-abdulmajidSustainablePalmOil2021}{Abdul Majid et
al., 2021}; \protect\hyperlink{ref-rspoWhoWeAre2023}{RSPO, 2023}). The
RSPO is formed by producers, retailers, investors and social and
environmental NGOs, with currently over 5000 members
(\protect\hyperlink{ref-rspoWhoWeAre2023}{RSPO, 2023}). With a share of
19\% on the oil palm market
(\protect\hyperlink{ref-ritchiePalmOil2021}{Ritchie, 2021}), it is
almost exclusively the only certification standard in global palm oil
trade (\protect\hyperlink{ref-murphyOilPalm2020s2021}{Murphy et al.,
2021}). Other labels such as the Malaysian Sustainable Palm Oil (MSPO)
and the Indonesian Sustainable Palm Oil (ISPO) have little to no share
in international markets and focus mainly on smallholders
(\protect\hyperlink{ref-murphyOilPalm2020s2021}{Murphy et al., 2021}).
RSPO states that compliance with its standards can mitigate the negative
impacts of palm oil production on the environment and local communities
(\protect\hyperlink{ref-rspoWhoWeAre2023}{2023}). Since 2005 RSPO
demands that new plantations must not have replaced primary forest and a
High Conservation Values (HCV) assesment is needed for certification,
which besides ecological factors, also considers a wide range of social
stakeholders, including local communities
(\protect\hyperlink{ref-murphyOilPalm2020s2021}{Murphy et al., 2021}).
Since 2018, a High Carbon Stock (HCS) report is additionally required,
which prohibits clearing for oil palm on HCS classified lands for
certification
(\protect\hyperlink{ref-rspoRSPOPrinciplesCriteria2018}{RSPO, 2018}).
Despite public discussion on oil palm, a study on Swiss consumers
showed, that only 9\% of the participants were even aware of the RSPO
label (\protect\hyperlink{ref-wassmannPalmOilRoundtable2023}{Wassmann et
al., 2023}).

A major criticism, however, is that if forests are not classified as HCV
or HCS RSPO certifications continue to allow logging
(\protect\hyperlink{ref-cazzollagattiSustainablePalmOil2019}{Cazzolla
Gatti et al., 2019}). In addition, significant tree losses were reported
prior to and post-certification
(\protect\hyperlink{ref-cazzollagattiSustainablePalmOil2019}{Cazzolla
Gatti et al., 2019}). Furthermore, deforestation rates in certified
areas are comparable to or even exceed those in uncertified areas,
leading Cazzolla Gatti et al.~to conclude that sustainable palm oil may
not be sustainable
(\protect\hyperlink{ref-cazzollagattiSustainablePalmOil2019}{2019}). The
criticism from non-governmental organizations is even more staggering.
They claim that RSPO auditors commit numerous violations in the
licensing process, such as failing to identify indigenous land rights
claims,faulty HCV assessments or serious labor abuses
(\protect\hyperlink{ref-eiaWhoWatchesWatchmen2015}{EIA, 2015};
\protect\hyperlink{ref-greenpeaceinternationalCertifyingDestruction2013}{Greenpeace
International, 2013}).

\hypertarget{other-crops}{%
\section{Other Crops}\label{other-crops}}

While some studies say that besides oil palm, which accounts for 88\% of
industrial plantations, pulp wood is the only crop to cover the other
12\% (\protect\hyperlink{ref-gaveauRiseFallForest2019}{Gaveau et al.,
2019}). However, other studies report that groundnut and coconut are
also grown in Borneo, but not in the from of industrial plantations
(\protect\hyperlink{ref-meijaardCoconutOilConservation2020}{Meijaard et
al., 2020}). Cocconut plantations occur mainly in smallholder form
(\textless4ha) and thus
(\protect\hyperlink{ref-meijaardCoconutOilConservation2020}{Meijaard et
al., 2020}), make it challenging to map with remote sensing methods
(\protect\hyperlink{ref-descalsHighresolutionGlobalMap2021}{Descals et
al., 2021}). However, it is thought to have almost a fivefold worse
impact on biodiversity
(\protect\hyperlink{ref-meijaardCoconutOilConservation2020}{Meijaard et
al., 2020}), These results, though, have caused a lot of controversy
(\protect\hyperlink{ref-rochmyaningsihClaimThatCoconut2020}{Rochmyaningsih,
2020}).

In Southeast Asia, the total cropland extent (excluding perennial woody
crops) saw only moderate growth. While 172.9 Mha were permanently used
as cropland between 2000 and 2003, 184.5 Mha were used between 2016 and
2019 (\protect\hyperlink{ref-potapovGlobalMapsCropland2021}{Potapov et
al., 2021}). A gain of 30.9 Mha offsetted a reduction of 19.3 Mha
(\protect\hyperlink{ref-potapovGlobalMapsCropland2021}{Potapov et al.,
2021}). Nevertheless, due to public awareness and the rapid expansion of
harvest area, oil palm has been extensively researched and mapped
(\protect\hyperlink{ref-descalsHighresolutionGlobalMap2021}{Descals et
al., 2021}), while maps on other crops harvest area distribution are
still sparse
(\protect\hyperlink{ref-meijaardCoconutOilConservation2020}{Meijaard et
al., 2020}).

In summary, due to much smaller expansion of cropland and perennial
woody crops, oil palm diminishes the importance of research on other
crops in Southeast Asia and thus there are hardly any maps available
making use of the advances in remoteley sensed data processing (see
Section~\ref{sec-remotesensing}).

\hypertarget{sec-infrastructure}{%
\section{Infrastructure}\label{sec-infrastructure}}

Expansion of infrastructure is happening at a high rate across the
globe. In the last two decades, the amount of built-up land area has
increased by 47\%, whereby this rate is even higher in Asia at 73\%
(\protect\hyperlink{ref-potapovGlobal20002020Land2022}{Potapov et al.,
2022}). With an additional 25 million kilometers of paved roads and more
than 300,000 kilometers of rail track, translating into a 60\% expansion
of land transportation infrastructure by 2050 compared to 2010,
infrastructure development will continue at a rapid rate
(\protect\hyperlink{ref-ieaGlobalLandTransport2013}{IEA, 2013};
\protect\hyperlink{ref-lauranceGlobalStrategyRoad2014}{Laurance et al.,
2014}). It is estimated that 90\% of this growth will take place in
developing countries projecting enourmous consequences for Borneo
(\protect\hyperlink{ref-lauranceGlobalStrategyRoad2014}{Laurance et al.,
2014}).

With deforestation and infrastructure development in remote areas,
hunting pressure is intensifying
(\protect\hyperlink{ref-kamyabElaeisGuineensis2022}{Kamyab, 2022}), as
well as poaching with improved accessibility
(\protect\hyperlink{ref-mooreAreRangerPatrols2018}{Moore et al., 2018}).
In addition, the expansion of habitation of humans stimulated by
infrastructure developments is leading to an increase in human-wildlife
conflicts (\protect\hyperlink{ref-kamyabElaeisGuineensis2022}{Kamyab,
2022}).

Not only does new infrastructure increase settlement and thus human
density, it is also associated with greater dispersal pressure of
invasive species, leading to new corridors such as roads or rail tracks
that facilitate their introduction
(\protect\hyperlink{ref-darRoadsActCorridors2015}{Dar et al., 2015};
\protect\hyperlink{ref-mungiRoleSpeciesRichness2021}{Mungi et al.,
2021}). Moreover, the resulting increase in light pollution affects the
behavior of individual animal species, which in turn can influence their
ecosystem services, eventually leading to effects of land erosion
(\protect\hyperlink{ref-lewanzikArtificialLightPuts2014}{Lewanzik \&
Voigt, 2014}).

Since large parts of Borneo are covered with forest, new infrastructure
projects cut through these areas, separating and isolating large
interconnected habitats
(\protect\hyperlink{ref-alamgirHighriskInfrastructureProjects2019}{Alamgir
et al., 2019}) and enhance deforestation nearby
(\protect\hyperlink{ref-meijerGlobalPatternsCurrent2018}{Meijer et al.,
2018};
\protect\hyperlink{ref-sloanHiddenChallengesConservation2019}{Sloan et
al., 2019}). Although ecologists agree that habitat loss has
far-reaching and detrimental consequences for biodiversity (see
Section~\ref{sec-deforestation}, Section~\ref{sec-oilpalm}), there is
disagreement about the impact that fragmentation itself has
(\protect\hyperlink{ref-didhamRethinkingConceptualFoundations2012a}{Didham
et al., 2012};
\protect\hyperlink{ref-fahrigRethinkingPatchSize2013}{Fahrig, 2013};
\protect\hyperlink{ref-haddadHabitatFragmentationIts2015a}{Haddad et
al., 2015};
\protect\hyperlink{ref-miller-rushingHowDoesHabitat2019}{Miller-Rushing
et al., 2019}). Observational studies often focused only on single
aspects of fragmentation (e.g., edge, isolation, area) and failed to
take in the full context of the complex, interconnected structures of
habitats
(\protect\hyperlink{ref-haddadHabitatFragmentationIts2015a}{Haddad et
al., 2015}). However, a comprehensive experiment indicated, that
biodiversity in fragmented habitats is reduced from 13\% up to 75\%
(\protect\hyperlink{ref-haddadHabitatFragmentationIts2015a}{Haddad et
al., 2015}). More recently, Püttker et al.~underlined that biodiversity
of forest-dependent animal and, to an even greater extent, plant species
are negatively affected by habitat fragmentation, especially through
edge effects
(\protect\hyperlink{ref-puttkerIndirectEffectsHabitat2020}{2020}). This
also applies to Borneo, as many infrastructure projects, including the
planned relocation of the Indonesian capital
(\protect\hyperlink{ref-lyonsWhyIndonesiaMoving2019}{Lyons, 2019}), are
estimated to have huge impacts on biodiversity
(\protect\hyperlink{ref-alamgirHighriskInfrastructureProjects2019}{Alamgir
et al., 2019}).

To sum up, the expansion of infrastructure contributes not only to
habitat loss, but also to increased habitat isolation due to
fragmentation, increased introduction of invasive species, and light
pollution, disturbing the fauna.

\hypertarget{sec-remotesensing}{%
\section{Advances in remote sensing}\label{sec-remotesensing}}

Since the launch of the Landsat-1 MSS satellite in 1972, an increasingly
comprehensive time series database has been built as newly launched
satellites have been equipped with more advanced technology
(\protect\hyperlink{ref-crowleyRemoteSensingRecent2020}{Crowley \&
Cardille, 2020}). However, it was not until 2008 that this data was made
available through the free and open Landsat data policy
(\protect\hyperlink{ref-zhuBenefitsFreeOpen2019}{Zhu et al., 2019}).
Combined with the launch of the European Space Agency's open-access
Copernicus mission, research from satellite data has exploded in recent
years (\protect\hyperlink{ref-crowleyRemoteSensingRecent2020}{Crowley \&
Cardille, 2020}; \protect\hyperlink{ref-zhuBenefitsFreeOpen2019}{Zhu et
al., 2019}). This development has also been facilitated by the
availability of massive-throughput analysis platforms like the Google
Earth Engine in 2010
(\protect\hyperlink{ref-crowleyRemoteSensingRecent2020}{Crowley \&
Cardille, 2020}). Another crucial achievement was the development of
machine learning algorithms that separates the remotely sensed data into
meaningful classifications
(\protect\hyperlink{ref-crowleyRemoteSensingRecent2020}{Crowley \&
Cardille, 2020}). Furthermore, machine learning offers unprecedented
potential for gaining deeper insights from historical data, as well as
filling in gaps in it
(\protect\hyperlink{ref-sarafanovMachineLearningApproach2020}{Sarafanov
et al., 2020}).

Despite these advancements, many high-resolution land cover and land use
(LCLU) datasets
(\protect\hyperlink{ref-descalsHighresolutionGlobalMap2021}{Descals et
al., 2021}; \protect\hyperlink{ref-karraGlobalLandUse2021}{Karra et al.,
2021}; \protect\hyperlink{ref-zanagaESAWorldCover102021}{Zanaga et al.,
2021}), while offering unprecedented detail, do not support
multi-decadal analysis
(\protect\hyperlink{ref-potapovGlobal20002020Land2022}{Potapov et al.,
2022}). However, medium resolution (30m resolution) multidecadal
datasets are more wideley available
(\protect\hyperlink{ref-danyloMapExtentYear2021}{Danylo et al., 2021};
\protect\hyperlink{ref-hansenHighResolutionGlobalMaps2013}{Hansen et
al., 2013};
\protect\hyperlink{ref-potapovGlobal20002020Land2022}{Potapov et al.,
2022}, \protect\hyperlink{ref-potapovGlobalMapsCropland2021}{2021};
\protect\hyperlink{ref-turubanovaOngoingPrimaryForest2018}{Turubanova et
al., 2018};
\protect\hyperlink{ref-tyukavinaGlobalTrendsForest2022}{Tyukavina et
al., 2022}).

\bookmarksetup{startatroot}

\hypertarget{sec-method}{%
\chapter{Method}\label{sec-method}}

\hypertarget{data-acquisition-and-selection}{%
\section{Data acquisition and
selection}\label{data-acquisition-and-selection}}

Initially, databases such as Web of Knowledge or Google Scholar were
searched for GIS and remote sensing studies covering the extent Borneo,
Southeast Asia or even a global extent and downloaded. QGIS (Version
3.30) was used for initial data exploration. This led to the manual
selection of data shown in Figure~\ref{fig-data_overview} for further
analysis. Hereafter, the datasets are referred to by the names specified
in the content column. The research area was defined as the landmass of
Borneo, explicitly excluding the smaller surrounding islands. This
delineation was obtained through the download of the Borneon boundary
using the OSMnx python package (Verison 1.3.0), which accesses the Open
Street Maps (OSM) database
(\protect\hyperlink{ref-boeingOSMnxPythonPackage2017}{Boeing, 2017}).

\begin{figure}

{\centering \includegraphics{text/05_method_files/data_overview.pdf}

}

\caption{\label{fig-data_overview}All datasets used in this thesis and
the temporal scope they span.}

\end{figure}

\hypertarget{software}{%
\section{Software}\label{software}}

All data preparation and analysis steps were performed using open-source
Python packages in Visual Studio Code (Version 1.83.0). Python version
was 3.10.9. Rasterio (Version 1.3.6) and NumPy (Version 1.24.3) were the
most relevant packages for data processing
(\protect\hyperlink{ref-gilliesRasterioDocumentation2023}{Gillies,
2023}; \protect\hyperlink{ref-harrisArrayProgrammingNumPy2020}{Harris et
al., 2020}). For each step a Python function was developed, which
requires an input path(s), optionally a target file path and an output
path. The output file was compressed in lzw form. The full code is
available on
\href{https://github.com/pfaffrob/03_vs_code}{github.com/pfaffrob/03\_vs\_code}.
OpenAI's Chat-GPT (versions 3.0, 3.5 and 4.0) was used to support code
generation. Functions were created using a minimal example with the help
of Chat-GPT. Subsequently, more complex features were implemented into
the function through personal adjustments or continuous user feedback to
the AI. The generated code was carefully inspected and reviewed to
ensure its correctness.

\hypertarget{data-preparation}{%
\section{Data preparation}\label{data-preparation}}

All data has been brought into a consistent format. This included steps
like merging multiple files, clipping to bbox, changing all values
outside the mask (e.g.~boundary of Borneo) to the \emph{nodata} value,
and reprojecting it to a uniform projection. For the analysis, the
Lambert Azimuthal Equal Area Projection was used with the center point
being at 115° longitude and 0° latitude. The workflow for the
preparation of raster and vector data is visible in
Figure~\ref{fig-data_preparation}.

\begin{figure}

\begin{minipage}[t]{\linewidth}

{\centering 

\raisebox{-\height}{

\includegraphics{text/05_method_files/raster_preparation.pdf}

}

}

\subcaption{\label{fig-raster_preparation}raster data preparation}
\end{minipage}%
\newline
\begin{minipage}[t]{\linewidth}

{\centering 

\raisebox{-\height}{

\includegraphics{text/05_method_files/shp_preparation.pdf}

}

}

\subcaption{\label{fig-shp_preparation}vector data preparation}
\end{minipage}%

\caption{\label{fig-data_preparation}Workflow of data preparation.
\emph{green: processing steps; yellow: files}}

\end{figure}

\begin{itemize}
\item
  Farbgebung vereinheitlichen (anpassen an
  Figure~\ref{fig-bardeforestation})
\item
  Detaillierter auf die verschiedenen Funktionen eingehen?
\end{itemize}

\hypertarget{analysis}{%
\section{Analysis}\label{analysis}}

For answering the simpler questions that required only one data set,
such as quantifying annual deforestation rates or total primary forest
area in 2000, the area calculations could be performed without any
further steps. For more complex questions, where multiple datasets were
involved, they were resampled onto the same grid. To create buffer zones
based on the built-up areas, each pixel whose center was located within
a radius of 100, 200, 500, 1000, and 2000 meters to a pixel of the
built-up area was used as a mask for further analysis. The workflow of
this is visible in Figure~\ref{fig-bufferworkflow}.

\begin{figure}

{\centering \includegraphics{text/05_method_files/buffer.pdf}

}

\caption{\label{fig-bufferworkflow}Workflow of buffer maps and
subsequent combination with other datasets}

\end{figure}

\begin{itemize}
\tightlist
\item
  Detaillierter auf die verschiedenen Funktionen eingehen?
\end{itemize}

\hypertarget{further-definitions}{%
\section{Further definitions}\label{further-definitions}}

The forest cover dataset shows the percentage of canopy cover. However,
Hansen et al.~provide no definition of what is considered forest
(\protect\hyperlink{ref-hansenHighResolutionGlobalMaps2013}{2013}).
Therefore, a threshold of 75\% closed canopy was chosen, which was also
used by Turubanova et al.~for comparison of their Results with GFW data
(\protect\hyperlink{ref-turubanovaOngoingPrimaryForest2018}{2018}).
However, Turubanov et al.~used expert interpreted training data for
detection of patterns recognized as primary forest
(\protect\hyperlink{ref-turubanovaOngoingPrimaryForest2018}{2018}).
Although areas with \textgreater75\% closed canopy overlap nearly all of
the primary forest areas (see Annex \textbf{XY}), these were
reclassified as forest (see Figure~\ref{fig-piefcover2000}).

Aggregated plants and perennial woody plants such as coconut and oil
palm were manually removed from subsequent analysis due to the cropland
definition in the used dataset
(\protect\hyperlink{ref-potapovGlobalMapsCropland2021}{Potapov et al.,
2021}), as well as crops with a total physical harvest area lower than
10000 ha within the project extent in any of the three datasets.

\begin{itemize}
\tightlist
\item
  Leider macht es trotz erfolgreicher Methodenentwicklung wenig Sinn
  dies genauer zu analysieren, da die intersection mit den Anbauflächen
  sehr gering ist. (Siehe Chapter~\ref{sec-discussion})
\end{itemize}

\bookmarksetup{startatroot}

\hypertarget{results}{%
\chapter{Results}\label{results}}

\hypertarget{forest-status-in-2000}{%
\section{Forest status in 2000}\label{forest-status-in-2000}}

At the beginning of the millennium, more than half (54.3\%) of the
island of Borneo, which has an area of 728,799 km\textsuperscript{2},
was covered with primary forests. The remaining areas were 30.4\%
covered with \textgreater= 75\% closed tree canopy, and 15.3\% with less
dense vegetation, respectively. Figure~\ref{fig-piefcover2000} provides
an overview of the composition of vegetation on Borneo at the start of
the study period.

\begin{figure}

{\centering \includegraphics[width=0.7\textwidth,height=\textheight]{text/../code/results/final_plots/fcover_2000.png}

}

\caption{\label{fig-piefcover2000}Composition of vegetated areas on
Borneo at the start of the study period.}

\end{figure}

The \textgreater= 75\% closed canopy consits of secondary forests as
well as plantations. 4886 km\textsuperscript{2} or 2.2\% of the
\textgreater75\% forest cover represent oil palms (stand 2000 or
detection up to 2003), with a lot of these oil palm plantations being
close to typical plantation patterns (see annex IV). Another notable
characteristic are large areas close to rivers, especially with human
settlements nearby, that have lot of dense vegetation right next to
primary forests (see annex V).

\hypertarget{deforestation}{%
\section{Deforestation}\label{deforestation}}

In the years from 2001 to 2021, a total of 16.81 Mha of deforestation
occured. 6.13 Mha of this is within primary forests and 10.68 Mha
outside of primary forests. In both, logging was mainly responsible for
forest loss, compared to forest fires. While logging rates in secondary
forests increased from 0.3 to almost 0.5 Mha per year by 2008, the five
most deforestation-intensive years, with 0.61 - 0.75 Mha of forest loss,
occurred between 2009 and 2016. After three years of steady
deforestation at 0.5 Mha per year, less than 0.4 Mha of forest loss was
recorded in 2020 and 2021 Figure~\ref{fig-bardeforestation}. Within
primary forests, logging rates increased from 0.11 Mha in 2002 to a
maximum of 0.50 in 2012 and declined back to 0.11 Mha in 2021
Figure~\ref{fig-bardeforestationprimary}. Although there was less
overall forest loss in primary forests, there were more forest fires
therein (0.81 Mha) than outside (0.72 Mha). Both primary and secondary
forests had their highest forest fire rates, by a large margin, in 2016
with 208,000 ha (secondary forest) and 255,000 ha (primary forest).
Other years with elevated forest fire rates were 2003, 2007, 2009, 2014
and 2015 ranging from 25,000 to 79,000 ha
Figure~\ref{fig-forestlossbarcharts}.

\begin{figure}

\begin{minipage}[t]{0.46\linewidth}

{\centering 

\raisebox{-\height}{

\includegraphics{text/../code/results/final_plots/total_deforestation.png}

}

}

\subcaption{\label{fig-bardeforestation}Secondary forest loss, peaking
in 2016.}
\end{minipage}%
%
\begin{minipage}[t]{0.09\linewidth}

{\centering 

~

}

\end{minipage}%
%
\begin{minipage}[t]{0.46\linewidth}

{\centering 

\raisebox{-\height}{

\includegraphics{text/../code/results/final_plots/total_primary_forest_deforestation.png}

}

}

\subcaption{\label{fig-bardeforestationprimary}Primary forest loss,
peaking in 2016.}
\end{minipage}%

\caption{\label{fig-forestlossbarcharts}Overview of deforestation within
\textbf{(a)} secondary forests and \textbf{(b)} primary forests.}

\end{figure}

\begin{figure}

{\centering \includegraphics{text/../code/results/maps/primary_deforestation.png}

}

\caption{\label{fig-map_deforestation_primary}Large areas in the
Malaysian part of Borneo where logging activities are penetrating deep
into the primary forest.}

\end{figure}

In general, logging patterns can be separated into two categories.
First, large areas of logging in the same year, often with sharp
borders, and second, dispersed small-scale logging
(Figure~\ref{fig-map_deforestation_op_fires},
Figure~\ref{fig-mapdeforestation}). This is especially the case in
primary forests in the malaysian part of Borneo, where logging
continously encroaches (Figure~\ref{fig-map_deforestation_primary}).
Fires in secondary forests have no obvious patterns; they occur randomly
and mainly on a small scale.

Protected areas cover 6.24 Mha, of which 0.08 Mha are secondary forest
and 4.72 Mha is primary forest. Within protected areas, the main reason
for forest loss were forest fires, regardless of the classification as
primary or secondary forest. Forest fire area and logging rates are
similar (Figure~\ref{fig-forestlossbarcharts_PA}). This is remarkable
given that secondary forests accounted for a massively smaller
proportion of the total area of PAs. Logging in protected areas occured
very limited with an average of \textasciitilde2,500 ha per year being
logged. However, wildfires occurred irregularly, with a major spike in
2016 and other substantial forest loss to fires in 2003, 2007, and 2015
(Figure~\ref{fig-bardeforestationprimary}). These majority of these
forest fires were on the edges of the forest spanning large patches
(Figure~\ref{fig-mapforestfires}).

\begin{figure}

\begin{minipage}[t]{0.50\linewidth}

{\centering 

\raisebox{-\height}{

\includegraphics{text/../code/results/final_plots/deforestation_protected_areas.png}

}

}

\subcaption{\label{fig-bardeforestationprotected}Annual rates of primary
forest loss within protected areas. The total area of primary forest in
these areas is 4.72 Mha.}
\end{minipage}%
%
\begin{minipage}[t]{0.50\linewidth}

{\centering 

\raisebox{-\height}{

\includegraphics{text/../code/results/final_plots/deforestation_primary_forest_protected_areas.png}

}

}

\subcaption{\label{fig-bardeforestationprotectedprimary}Annual rates of
secondary forest loss within protected areas. The total area of primary
forest in these areas is 0.08 Mha.}
\end{minipage}%

\caption{\label{fig-forestlossbarcharts_PA}Deforestation within
protected areas, with the peak year in 2016.}

\end{figure}

\begin{figure}

{\centering \includegraphics[width=1\textwidth,height=\textheight]{text/../code/results/maps/deforestation_protected_areas_forest_fires.png}

}

\caption{\label{fig-mapforestfires}In this protected area it is clearly
visible, that forest fires (red) are the main drivers of deforestation
within protected areas. Large forest fires like these are the cause of
the outlier year 2016.}

\end{figure}

\hypertarget{oil-palm}{%
\section{Oil Palm}\label{oil-palm}}

\begin{figure}

{\centering \includegraphics{text/../code/results/maps/deforestation_op_fires.png}

}

\caption{\label{fig-map_deforestation_op_fires}Large deforestation areas
are found on and nearby newly detected oilpalm data. This extract also
shows one of the largest oil palm plantations on previous primary
forest.}

\end{figure}

\hypertarget{infrastructure}{%
\section{Infrastructure}\label{infrastructure}}

\begin{figure}

\begin{minipage}[t]{0.50\linewidth}

{\centering 

\raisebox{-\height}{

\includegraphics{text/../code/results/final_plots/pie_existing_built_up.png}

}

}

\subcaption{\label{fig-forestloss_pie_existing_built_up}Location of
existing built-up areas in 2000.}
\end{minipage}%
%
\begin{minipage}[t]{0.50\linewidth}

{\centering 

\raisebox{-\height}{

\includegraphics{text/../code/results/final_plots/pie_new_built_up.png}

}

}

\subcaption{\label{fig-forestloss_pie_new_built_up}Location of new
built-up areas in the period between 2001 and 2020..}
\end{minipage}%

\caption{\label{fig-forestloss_built_up}The built-up area within the
study period has more than doubled to a total of 21,789
km\textsuperscript{2}.}

\end{figure}

The built up areas have increased by 110\% from 10,332
km\textsuperscript{2} in 2000 to a total of 21,729~km\textsuperscript{2}
in 2020. There was only modest overlap (0.8\%) with forest loss in
primary forests from built up areas in 2000, and more substantial
(17.7\%) with other dense vegetation areas. Looking at the maps, these
can be attributed mainly to maintenance work and the corresponding
cutting back of adjacent vegetation, as well on a smaller scale to the
clearing of oil palm plantations for replanting. Annex III provides an
representative example.

\begin{figure}

{\centering \includegraphics{text/../code/results/final_plots/deforestation_buffer_primary_forest.png}

}

\caption{\label{fig-primary_buffer}The x-axis shows the primary forest
area that falls within the buffer distance. The y-axis shows the amount
of primary forest loss. This shows the relationship of infrastructure
proximity for \textbf{a)} total built-up area 2020 (assumed that no year
2000 areas were removed), \textbf{b)} built-up area in the year 2000 and
\textbf{c)} newly built-up areas between 2000 and 2020.}

\end{figure}

\begin{figure}

{\centering \includegraphics{text/../code/results/final_plots/op_buffer.png}

}

\caption{\label{fig-op_buffer}Composition of vegetated areas on Borneo
at the start of the study period.}

\end{figure}

\begin{figure}

{\centering \includegraphics[width=1\textwidth,height=\textheight]{text/../code/results/maps/deforestation_patterns.png}

}

\caption{\label{fig-mapdeforestation}Typical deforestation patterns in
eastern Borneo with a few large areas of deforestation inland, smaller
scale near the coast, and a clear delineation to protected area with
very little forest loss.}

\end{figure}

\hypertarget{tbl-buffer_primary}{}
\begin{longtable}[]{@{}
  >{\raggedleft\arraybackslash}p{(\columnwidth - 24\tabcolsep) * \real{0.0667}}
  >{\raggedright\arraybackslash}p{(\columnwidth - 24\tabcolsep) * \real{0.0167}}
  >{\centering\arraybackslash}p{(\columnwidth - 24\tabcolsep) * \real{0.1111}}
  >{\centering\arraybackslash}p{(\columnwidth - 24\tabcolsep) * \real{0.0944}}
  >{\centering\arraybackslash}p{(\columnwidth - 24\tabcolsep) * \real{0.0889}}
  >{\centering\arraybackslash}p{(\columnwidth - 24\tabcolsep) * \real{0.0167}}
  >{\centering\arraybackslash}p{(\columnwidth - 24\tabcolsep) * \real{0.1111}}
  >{\centering\arraybackslash}p{(\columnwidth - 24\tabcolsep) * \real{0.0944}}
  >{\centering\arraybackslash}p{(\columnwidth - 24\tabcolsep) * \real{0.0889}}
  >{\centering\arraybackslash}p{(\columnwidth - 24\tabcolsep) * \real{0.0167}}
  >{\centering\arraybackslash}p{(\columnwidth - 24\tabcolsep) * \real{0.1111}}
  >{\centering\arraybackslash}p{(\columnwidth - 24\tabcolsep) * \real{0.0944}}
  >{\centering\arraybackslash}p{(\columnwidth - 24\tabcolsep) * \real{0.0889}}@{}}
\caption{\label{tbl-buffer_primary}The percentages of primary forest
loss are calculated within the confines of the primary forest that lies
within certain buffer distances. The exposure of the primary forest to
deforestation intensifies as its proximity to infrastructure
increases.}\tabularnewline
\toprule\noalign{}
\begin{minipage}[b]{\linewidth}\raggedleft
\textbf{Buffer}
\end{minipage} & \begin{minipage}[b]{\linewidth}\raggedright
\end{minipage} & \begin{minipage}[b]{\linewidth}\centering
\end{minipage} & \begin{minipage}[b]{\linewidth}\centering
\textbf{total}
\end{minipage} & \begin{minipage}[b]{\linewidth}\centering
\end{minipage} & \begin{minipage}[b]{\linewidth}\centering
\end{minipage} & \begin{minipage}[b]{\linewidth}\centering
\end{minipage} & \begin{minipage}[b]{\linewidth}\centering
\textbf{existing}
\end{minipage} & \begin{minipage}[b]{\linewidth}\centering
\end{minipage} & \begin{minipage}[b]{\linewidth}\centering
\end{minipage} & \begin{minipage}[b]{\linewidth}\centering
\end{minipage} & \begin{minipage}[b]{\linewidth}\centering
\textbf{new}
\end{minipage} & \begin{minipage}[b]{\linewidth}\centering
\end{minipage} \\
\midrule\noalign{}
\endfirsthead
\toprule\noalign{}
\begin{minipage}[b]{\linewidth}\raggedleft
\textbf{Buffer}
\end{minipage} & \begin{minipage}[b]{\linewidth}\raggedright
\end{minipage} & \begin{minipage}[b]{\linewidth}\centering
\end{minipage} & \begin{minipage}[b]{\linewidth}\centering
\textbf{total}
\end{minipage} & \begin{minipage}[b]{\linewidth}\centering
\end{minipage} & \begin{minipage}[b]{\linewidth}\centering
\end{minipage} & \begin{minipage}[b]{\linewidth}\centering
\end{minipage} & \begin{minipage}[b]{\linewidth}\centering
\textbf{existing}
\end{minipage} & \begin{minipage}[b]{\linewidth}\centering
\end{minipage} & \begin{minipage}[b]{\linewidth}\centering
\end{minipage} & \begin{minipage}[b]{\linewidth}\centering
\end{minipage} & \begin{minipage}[b]{\linewidth}\centering
\textbf{new}
\end{minipage} & \begin{minipage}[b]{\linewidth}\centering
\end{minipage} \\
\midrule\noalign{}
\endhead
\bottomrule\noalign{}
\endlastfoot
& & \textbf{primary forest} & \textbf{loss buffer} & \textbf{loss total}
& & \textbf{primary forest} & \textbf{loss buffer} & \textbf{loss total}
& & \textbf{primary forest} & \textbf{loss buffer} & \textbf{loss
total} \\
& & & & & & & & & & & & \\
\textbf{100m} & & 4.6 & 71.4 & 21.4 & & 0.9 & 44.4 & 2.5 & & 4.6 & 66.7
& 19.9 \\
\textbf{200m} & & 10.6 & 58.1 & 39.6 & & 3.3 & 38.3 & 8.1 & & 10.6 &
53.2 & 36.3 \\
\textbf{500m} & & 24.4 & 42.4 & 66.7 & & 11.4 & 33.4 & 24.7 & & 24.4 &
39.3 & 61.9 \\
\textbf{1000m} & & 39.6 & 32.5 & 82.7 & & 23.5 & 29.7 & 45.0 & & 39.4 &
31.0 & 78.8 \\
\textbf{2000m} & & 57.3 & 25.2 & 93.0 & & 41.5 & 26.0 & 69.5 & & 57.3 &
24.7 & 91.2 \\
\end{longtable}

\hypertarget{tbl-buffer_op}{}
\begin{longtable}[]{@{}
  >{\raggedleft\arraybackslash}p{(\columnwidth - 24\tabcolsep) * \real{0.0667}}
  >{\raggedright\arraybackslash}p{(\columnwidth - 24\tabcolsep) * \real{0.0167}}
  >{\centering\arraybackslash}p{(\columnwidth - 24\tabcolsep) * \real{0.1111}}
  >{\centering\arraybackslash}p{(\columnwidth - 24\tabcolsep) * \real{0.1056}}
  >{\centering\arraybackslash}p{(\columnwidth - 24\tabcolsep) * \real{0.0778}}
  >{\centering\arraybackslash}p{(\columnwidth - 24\tabcolsep) * \real{0.0167}}
  >{\centering\arraybackslash}p{(\columnwidth - 24\tabcolsep) * \real{0.1111}}
  >{\centering\arraybackslash}p{(\columnwidth - 24\tabcolsep) * \real{0.1056}}
  >{\centering\arraybackslash}p{(\columnwidth - 24\tabcolsep) * \real{0.0778}}
  >{\centering\arraybackslash}p{(\columnwidth - 24\tabcolsep) * \real{0.0167}}
  >{\centering\arraybackslash}p{(\columnwidth - 24\tabcolsep) * \real{0.1111}}
  >{\centering\arraybackslash}p{(\columnwidth - 24\tabcolsep) * \real{0.1056}}
  >{\centering\arraybackslash}p{(\columnwidth - 24\tabcolsep) * \real{0.0778}}@{}}
\caption{\label{tbl-buffer_op}Percentages of new and existing oil palm
(OP) plantations within buffer distances from the total area of Borneo.
The data is categorized into three sections: area of Borneo (portion of
the total Bornean Area covered by the buffer), new OP buffer (portion
newly detected oil palm from 2001 - 2017 within buffer area), and total
OP (portion newly detected oil palm from 2001 - 2017 from all new oil
palm area). This is done for \textbf{i)} \emph{total} infrastructure in
2020 (assuming no year 2000 built-up area was removed), \textbf{ii)}
\emph{existing} (year 2000) built up area and \textbf{iii)} \emph{new}
built-up area (2001 - 2020).}\tabularnewline
\toprule\noalign{}
\begin{minipage}[b]{\linewidth}\raggedleft
\textbf{Buffer}
\end{minipage} & \begin{minipage}[b]{\linewidth}\raggedright
\end{minipage} & \begin{minipage}[b]{\linewidth}\centering
\end{minipage} & \begin{minipage}[b]{\linewidth}\centering
\textbf{total}
\end{minipage} & \begin{minipage}[b]{\linewidth}\centering
\end{minipage} & \begin{minipage}[b]{\linewidth}\centering
\end{minipage} & \begin{minipage}[b]{\linewidth}\centering
\end{minipage} & \begin{minipage}[b]{\linewidth}\centering
\textbf{existing}
\end{minipage} & \begin{minipage}[b]{\linewidth}\centering
\end{minipage} & \begin{minipage}[b]{\linewidth}\centering
\end{minipage} & \begin{minipage}[b]{\linewidth}\centering
\end{minipage} & \begin{minipage}[b]{\linewidth}\centering
\textbf{new}
\end{minipage} & \begin{minipage}[b]{\linewidth}\centering
\end{minipage} \\
\midrule\noalign{}
\endfirsthead
\toprule\noalign{}
\begin{minipage}[b]{\linewidth}\raggedleft
\textbf{Buffer}
\end{minipage} & \begin{minipage}[b]{\linewidth}\raggedright
\end{minipage} & \begin{minipage}[b]{\linewidth}\centering
\end{minipage} & \begin{minipage}[b]{\linewidth}\centering
\textbf{total}
\end{minipage} & \begin{minipage}[b]{\linewidth}\centering
\end{minipage} & \begin{minipage}[b]{\linewidth}\centering
\end{minipage} & \begin{minipage}[b]{\linewidth}\centering
\end{minipage} & \begin{minipage}[b]{\linewidth}\centering
\textbf{existing}
\end{minipage} & \begin{minipage}[b]{\linewidth}\centering
\end{minipage} & \begin{minipage}[b]{\linewidth}\centering
\end{minipage} & \begin{minipage}[b]{\linewidth}\centering
\end{minipage} & \begin{minipage}[b]{\linewidth}\centering
\textbf{new}
\end{minipage} & \begin{minipage}[b]{\linewidth}\centering
\end{minipage} \\
\midrule\noalign{}
\endhead
\bottomrule\noalign{}
\endlastfoot
& & \textbf{area of Borneo} & \textbf{new OP buffer} & \textbf{total OP}
& & \textbf{area of Borneo} & \textbf{new OP buffer} & \textbf{total OP}
& & \textbf{area of Borneo} & \textbf{new OP buffer} & \textbf{total
OP} \\
& & & & & & & & & & & & \\
\textbf{100m} & & 11.0 & 15.5 & 31.7 & & 5.1 & 11.4 & 10.7 & & 7.8 &
16.9 & 24.6 \\
\textbf{200m} & & 21.2 & 14.6 & 57.7 & & 10.5 & 11.4 & 22.2 & & 16.6 &
15.6 & 48.0 \\
\textbf{500m} & & 41.2 & 11.7 & 89.8 & & 24.5 & 10.6 & 48.4 & & 35.2 &
12.4 & 81.6 \\
\textbf{1000m} & & 57.9 & 9.2 & 98.8 & & 41.1 & 9.4 & 71.6 & & 52.6 &
9.8 & 96.1 \\
\textbf{2000m} & & 73.3 & 7.3 & 99.9 & & 60.6 & 8.0 & 90.3 & & 69.3 &
7.7 & 99.7 \\
\end{longtable}

Table~\ref{tbl-buffer_op}
Table~\ref{tbl-buffer_op}, table~\ref{tbl-buffer_op}

Additional map extracts are provided in Appendix 3. For best data
exploration, visit \emph{link to storymap}.

\bookmarksetup{startatroot}

\hypertarget{sec-discussion}{%
\chapter{Discussion}\label{sec-discussion}}

However, the data showing forest gain in the span of 2001 - 2012 is
inaccurate. Analysis showed that 32\% of the alleged forest gain
represents newly established palm oil plantations. Thus, all new palm
oil cultivation areas established between 2001 and 2012 have been
removed from the forest gain dataset.

\hypertarget{data-selection}{%
\section{Data selection}\label{data-selection}}

\hypertarget{crop-harvest-area}{%
\subsection{Crop harvest area}\label{crop-harvest-area}}

Existing mapping of cropland all take a similar approach. They
disaggregate national and subnational statistics and estimate their
allocation to low-resolution grid cells of 5 arcminutes (Wood-Sichra et
al.
(\protect\hyperlink{ref-wood-sichraSpatialProductionAllocation2016}{2016});
You et al.
(\protect\hyperlink{ref-youGeneratingPlausibleCrop2009}{2009}); Yu et
al. (\protect\hyperlink{ref-yuCultivatedPlanet20102020}{2020}); Grogan
et al. (\protect\hyperlink{ref-groganGlobalGriddedCrop2022}{2022})).
Among them, the Spatial Production Allocation Model (SPAM) takes the
most factors into account rendering it the most comprehensive data. (You
\& Sun (\protect\hyperlink{ref-youMappingGlobalCropping2022}{2022}))
Additionally, the dataset family currently contains data for the years
2000, 2005 and 2010, which makes it the most suited choice for the
present work (Wood-Sichra et al.
(\protect\hyperlink{ref-wood-sichraSpatialProductionAllocation2016}{2016});
You et al.
(\protect\hyperlink{ref-youGeneratingPlausibleCrop2009}{2009}); Yu et
al. (\protect\hyperlink{ref-yuCultivatedPlanet20102020}{2020})).
However, due to improvements in the method of creating the datasets, the
more recent ones are a more accurate representation of the physical
harvest area and are more finely broken down into distinct crops.
(Wood-Sichra et al.
(\protect\hyperlink{ref-wood-sichraSpatialProductionAllocation2016}{2016});
Yu et al. (\protect\hyperlink{ref-yuCultivatedPlanet20102020}{2020})).
This is why coconut is missing in the year 2000 even though it was an
important crop for both Indonesia and Malaysia at that time
(\protect\hyperlink{ref-faoFAOSTATDatabase2023}{FAO, 2023}).

Although a more recent dataset called Global Agro-ecological Zones (GAEZ
v4.0) was available for harvested crops in 2015, no analysis was
conducted with it because the data lacked the necessary quality and
therefore could not be compared to the SPAM datasets. Just like the SPAM
dataset, FAO statistics complemented with national data on FAO gaps,
were used to calculate crop yield in a 5 arcminute resolution grid.
However, the pixel values in the GAEZ 2015 dataset are based upon the
GAEZ 2010 (GAEZ v3.0) dataset, whose download portal does not work
anymore (FAO/IIASA
(\protect\hyperlink{ref-faoux2fiiasaGlobalAgroecologicalZones2010}{2010})).
It was calculated by multiplying the year GAEZ 2010 pixel value by the
countries' change in the given crop over the last 5 years. As a result,
any spatial displacement in the harvest area of a given crop to another
region is incorrectly mapped rendering the intended analysis inadequate.

Another available dataset for 2000 was neglected because the SPAM
datasets were all created using the same method, which reduces
inconsistency when comparing spatiotemporal differences. In addition,
only the harvested area was available in the alternative dataset. This
means that areas that were harvested several times per year were also
counted multiple times, biasing the total physically cultivated area.

Nevertheless, SPAM data is also subject to non-negligible drawbacks,
since \#\#\# !!!NOCH ERGÄNZEN!!! \#\#\# (Joglekar et al.
(\protect\hyperlink{ref-joglekarPixelatingCropProduction2019}{2019})).

\hypertarget{processing}{%
\section{Processing}\label{processing}}

For area calculations, it is crucial to choose a well-suited coordinate
reference system. Taking a UTM zone was considered too inaccurate, as
Borneo spans over four UTM zones. The only crs available covering Borneo
is the Timbalai 1948
(\protect\hyperlink{ref-klokantechnologiesgmbhTimbalai1948RSO}{Klokan
Technologies GmbH, n.d.}). But even this projection only adequately
depicts one part, or more precisely the Malaysian part, and thus
distorts the larger Indonesian part of Borneo. Therefore the Lambert
Azimuthal Equal Area Projection was chosen since it is well suited for
accurately showing the area in large spatial
extents.(\protect\hyperlink{ref-esriQuick_Notes_on_Map_Projections_in_ArcGIS_nov2019Pdf2019}{esri,
2019}, \protect\hyperlink{ref-esriLambertAzimuthalEqualarea2023}{2023}).
Although this does not affect the area calculations, the center was
manually set to the center of Borneo to achieve a visualization of
Borneo that minimizes distortion. As indicated before, merging data from
different sources is challenging. This is especially true for the
croplands in higher resolution (\textasciitilde30 m) with the physical
harvest area for the individual crops in lower resolution (5
arcminutes). Especially because the two datasets were created with
fundamentally different methods. While the dataset of the cultivated
areas was created by analysis of satellite data time-series and showing
their precise extent (Potapov et al.
(\protect\hyperlink{ref-potapovGlobalMapsCropland2021}{2021})), the
physical harvested area maps of the different crops were by a complex
model based on statistics and estimates (Wood-Sichra et al.
(\protect\hyperlink{ref-wood-sichraSpatialProductionAllocation2016}{2016});
You et al.
(\protect\hyperlink{ref-youGeneratingPlausibleCrop2009}{2009}); Yu et
al. (\protect\hyperlink{ref-yuCultivatedPlanet20102020}{2020})). Even
though a method to aggregate these datasets was developed, the main
reason why crops were not further analyzed was that the intersection
area of newly detected oil palm plantations was below a total of 80 km²
for all of Borneo over the span from 2000 to 2017. Since this represents
only such a small fraction of the total cropland (\textasciitilde2000 -
\textasciitilde3500 km²) and

\hypertarget{forest}{%
\section{Forest}\label{forest}}

One reason primary forests do not extend to the riverbanks is that the
river helps shape the surrounding forest through bank shifting or
flooding. Much more likely, however, is that the rivers serve as
transportation routes and thus provide easy access for logging. This
assumption is also supported by the fact that logging occurred adjacent
to the river during the study period (see Annex V).

It is important to note, that the forest cover dataset only represents
tree crowns. This means, that

Large interconnected forests, including primary forest still exist on
Borne\ldots{} Central Borneo possesses one of the last interconnected
large forest fragments, rendering it especially important to protect it
(\protect\hyperlink{ref-haddadHabitatFragmentationIts2015a}{Haddad et
al., 2015}).

hansen et al weist auf erster seite hin, dass indonesiische regierung
2011 neue regulierungen eingeführt hat zu abholzung.

Due to the method applied for forest loss detection, there was also some
deforestation outside secondary forest or dense vegetation.

\hypertarget{infrastructure-1}{%
\section{Infrastructure}\label{infrastructure-1}}

Theoretically, there should not be any deforestation on built up areas
from 2000. While the marginal 0.8\% overlap with primary forest clearing
confirms the accuracy of the data, the 17.7\% with residual dense
vegetation cannot be attributed exclusively to this cause (see
Figure~\ref{fig-forestloss_pie_existing_built_up}).

\hypertarget{oil-palm-1}{%
\subsection{Oil Palm}\label{oil-palm-1}}

Whilst this thesis allocates \textasciitilde4 Mha to newly detected oil
palm plantations, another paper states a higher number with
\textasciitilde{} 5.5 Mha (Gaveau et al.
(\protect\hyperlink{ref-gaveauRiseFallForest2019}{2019})).
--\textgreater{} different crs (crs not given), falsely classified to
pulp wood as difficult to distinguish.

However, it would be too simple to blame this deforestation only to palm
oil as the single driver as it remains unknown whether oil palm caused
or simply followed deforestation
(\protect\hyperlink{ref-fitzherbertHowWillOil2008}{Fitzherbert et al.,
2008}). Companies use palm oil concessions to obtain permission to log,
and thus make money, without cultivating palm oil on the land
(\protect\hyperlink{ref-fitzherbertHowWillOil2008}{Fitzherbert et al.,
2008}).

Even though plantation expansion has slowed, the recent large price jump
(from US\$601 per ton in 2019 to US\$1276 in 2022) is a worrying
indicator (\protect\hyperlink{ref-worldbankAveragePricesPalm2023}{World
Bank, 2023}). This because the price is closely linked to the expansion
of oil palm production
(\protect\hyperlink{ref-gaveauSlowingDeforestationIndonesia2022}{Gaveau
et al., 2022}) and thus to environmental issues discussed in
Section~\ref{sec-oilpalm}.

\hypertarget{biodiversity}{%
\subsection{Biodiversity}\label{biodiversity}}

If the deforestation drivers are not addressed, biodiversity will
continue to decline, as examined in a case study of the Borneo orangutan
(Voigt et al.
(\protect\hyperlink{ref-voigtDeforestationProjectionsImply2022}{2022})).

\emph{Our results show the importance of protecting forests from
anthropogenic modifications. Protection of hyperdiverse tropical forests
results in a positive feedback, since they reduce biological invasions,
which will in turn safeguard native species richness and abundance,
ensuring their continued ecosystem services (IPBES, 2019b).}
(\protect\hyperlink{ref-mungiRoleSpeciesRichness2021}{Mungi et al.,
2021})

\bookmarksetup{startatroot}

\hypertarget{references}{%
\chapter{References}\label{references}}

\hypertarget{refs}{}
\begin{CSLReferences}{1}{0}
\leavevmode\vadjust pre{\hypertarget{ref-abdulmajidSustainablePalmOil2021}{}}%
Abdul Majid, N., Ramli, Z., Md Sum, S., \& Awang, A. H. (2021).
Sustainable {Palm Oil Certification Scheme Frameworks} and {Impacts}: {A
Systematic Literature Review}. \emph{Sustainability}, \emph{13}(6),
3263. \url{https://doi.org/10.3390/su13063263}

\leavevmode\vadjust pre{\hypertarget{ref-alamgirHighriskInfrastructureProjects2019}{}}%
Alamgir, M., Campbell, M. J., Sloan, S., Suhardiman, A., Supriatna, J.,
\& Laurance, W. F. (2019). High-risk infrastructure projects pose
imminent threats to forests in {Indonesian Borneo}. \emph{Scientific
Reports}, \emph{9}(1), 140.
\url{https://doi.org/10.1038/s41598-018-36594-8}

\leavevmode\vadjust pre{\hypertarget{ref-austinWhatCausesDeforestation2019}{}}%
Austin, K. G., Schwantes, A., Gu, Y., \& Kasibhatla, P. S. (2019). What
causes deforestation in {Indonesia}? \emph{Environmental Research
Letters}, \emph{14}(2), 024007.
\url{https://doi.org/10.1088/1748-9326/aaf6db}

\leavevmode\vadjust pre{\hypertarget{ref-barnesConsequencesTropicalLand2014}{}}%
Barnes, A. D., Jochum, M., Mumme, S., Haneda, N. F., Farajallah, A.,
Widarto, T. H., \& Brose, U. (2014). Consequences of tropical land use
for multitrophic biodiversity and ecosystem functioning. \emph{Nature
Communications}, \emph{5}(1), 5351.
\url{https://doi.org/10.1038/ncomms6351}

\leavevmode\vadjust pre{\hypertarget{ref-basironOILPALMITS2004}{}}%
Basiron, Y., \& Weng, C. K. (2004). {THE OIL PALM AND ITS
SUSTAINABILITY}. \emph{JOURNAL OF OIL PALM RESEARCH}.

\leavevmode\vadjust pre{\hypertarget{ref-boeingOSMnxPythonPackage2017}{}}%
Boeing, G. (2017). \emph{{OSMnx}: {A Python} package to work with
graph-theoretic {OpenStreetMap} street networks}.
\url{https://doi.org/DOI:10.21105/joss.00215}

\leavevmode\vadjust pre{\hypertarget{ref-buschReductionsEmissionsDeforestation2015}{}}%
Busch, J., Ferretti-Gallon, K., Engelmann, J., Wright, M., Austin, K.
G., Stolle, F., Turubanova, S., Potapov, P. V., Margono, B., Hansen, M.
C., \& Baccini, A. (2015). Reductions in emissions from deforestation
from {Indonesia}'s moratorium on new oil palm, timber, and logging
concessions. \emph{Proceedings of the National Academy of Sciences},
\emph{112}(5), 1328--1333. \url{https://doi.org/10.1073/pnas.1412514112}

\leavevmode\vadjust pre{\hypertarget{ref-byerleeTropicalOilCrop2016}{}}%
Byerlee, D., Falcon, W. P., \& Naylor, R. L. (2016). \emph{The {Tropical
Oil Crop Revolution}: {Food}, {Feed}, {Fuel}, and {Forests}}. {Oxford
University Press}.
\url{https://doi.org/10.1093/acprof:oso/9780190222987.001.0001}

\leavevmode\vadjust pre{\hypertarget{ref-cazzollagattiSustainablePalmOil2019}{}}%
Cazzolla Gatti, R., Liang, J., Velichevskaya, A., \& Zhou, M. (2019).
Sustainable palm oil may not be so sustainable. \emph{Science of The
Total Environment}, \emph{652}, 48--51.
\url{https://doi.org/10.1016/j.scitotenv.2018.10.222}

\leavevmode\vadjust pre{\hypertarget{ref-changEconomicPerspectiveOil2003}{}}%
Chang, E. S., Abdul Rahim, A. S., \& Zainon, B. (2003). \emph{An
{Economic Perspective} of {Oil Extraction Rate} in the {Oil Palm
Industry} of {Malaysia}}. \emph{3}(1), 25--31.

\leavevmode\vadjust pre{\hypertarget{ref-chewImprovingSustainabilityPalm2021}{}}%
Chew, C. L., Ng, C. Y., Hong, W. O., Wu, T. Y., Lee, Y.-Y., Low, L. E.,
Kong, P. S., \& Chan, E. S. (2021). Improving {Sustainability} of {Palm
Oil Production} by {Increasing Oil Extraction Rate}: A {Review}.
\emph{Food and Bioprocess Technology}, \emph{14}(4), 573--586.
\url{https://doi.org/10.1007/s11947-020-02555-1}

\leavevmode\vadjust pre{\hypertarget{ref-coetzeeOpenGeospatialSoftware2020}{}}%
Coetzee, S., Ivánová, I., Mitasova, H., \& Brovelli, M. (2020). Open
{Geospatial Software} and {Data}: {A Review} of the {Current State} and
{A Perspective} into the {Future}. \emph{ISPRS International Journal of
Geo-Information}, \emph{9}(2), 90.
\url{https://doi.org/10.3390/ijgi9020090}

\leavevmode\vadjust pre{\hypertarget{ref-crowleyRemoteSensingRecent2020}{}}%
Crowley, M. A., \& Cardille, J. A. (2020). Remote {Sensing}'s {Recent}
and {Future Contributions} to {Landscape Ecology}. \emph{Current
Landscape Ecology Reports}, \emph{5}(3), 45--57.
\url{https://doi.org/10.1007/s40823-020-00054-9}

\leavevmode\vadjust pre{\hypertarget{ref-curtisClassifyingDriversGlobal2018}{}}%
Curtis, P. G., Slay, C. M., Harris, N. L., Tyukavina, A., \& Hansen, M.
C. (2018). Classifying drivers of global forest loss. \emph{Science},
\emph{361}(6407), 1108--1111.
\url{https://doi.org/10.1126/science.aau3445}

\leavevmode\vadjust pre{\hypertarget{ref-danyloMapExtentYear2021}{}}%
Danylo, O., Pirker, J., Lemoine, G., Ceccherini, G., See, L., McCallum,
I., Hadi, Kraxner, F., Achard, F., \& Fritz, S. (2021). A map of the
extent and year of detection of oil palm plantations in {Indonesia},
{Malaysia} and {Thailand}. \emph{Scientific Data}, \emph{8}(1), 96.
\url{https://doi.org/10.1038/s41597-021-00867-1}

\leavevmode\vadjust pre{\hypertarget{ref-darRoadsActCorridors2015}{}}%
Dar, P. A., Reshi, Z. A., \& Shah, M. A. (2015). \emph{Roads act as
corridors for the spread of alien plant species in the mountainous
regions: {A} case study of {Kashmir Valley}, {India}}.

\leavevmode\vadjust pre{\hypertarget{ref-descalsHighresolutionGlobalMap2021}{}}%
Descals, A., Wich, S., Meijaard, E., Gaveau, D. L. A., Peedell, S., \&
Szantoi, Z. (2021). High-resolution global map of smallholder and
industrial closed-canopy oil palm plantations. \emph{Earth System
Science Data}, \emph{13}(3), 1211--1231.
\url{https://doi.org/10.5194/essd-13-1211-2021}

\leavevmode\vadjust pre{\hypertarget{ref-didhamRethinkingConceptualFoundations2012a}{}}%
Didham, R. K., Kapos, V., \& Ewers, R. M. (2012). Rethinking the
conceptual foundations of habitat fragmentation research. \emph{Oikos},
\emph{121}(2), 161--170.
\url{https://doi.org/10.1111/j.1600-0706.2011.20273.x}

\leavevmode\vadjust pre{\hypertarget{ref-eiaWhoWatchesWatchmen2015}{}}%
EIA. (2015). \emph{Who watches the watchmen?}

\leavevmode\vadjust pre{\hypertarget{ref-esriQuick_Notes_on_Map_Projections_in_ArcGIS_nov2019Pdf2019}{}}%
esri. (2019).
\emph{Quick\_{Notes}\_on\_{Map}\_{Projections}\_in\_{ArcGIS}\_nov2019.pdf}.

\leavevmode\vadjust pre{\hypertarget{ref-esriLambertAzimuthalEqualarea2023}{}}%
esri. (2023). \emph{Lambert azimuthal
equal-area\textemdash{{ArcGIS Pro}} \textbar{} {Documentation}}.
https://pro.arcgis.com/en/pro-app/latest/help/mapping/properties/lambert-azimuthal-equal-area.htm.

\leavevmode\vadjust pre{\hypertarget{ref-fahrigRethinkingPatchSize2013}{}}%
Fahrig, L. (2013). Rethinking patch size and isolation effects: The
habitat amount hypothesis. \emph{Journal of Biogeography}, \emph{40}(9),
1649--1663. \url{https://doi.org/10.1111/jbi.12130}

\leavevmode\vadjust pre{\hypertarget{ref-faoGlobalForestResources2020}{}}%
FAO. (2020). \emph{Global {Forest Resources Assessment} 2020}. {FAO}.
\url{https://doi.org/10.4060/ca9825en}

\leavevmode\vadjust pre{\hypertarget{ref-faoFAOSTATDatabase2023}{}}%
FAO. (2023). \emph{{FAOSTAT Database}}.
https://www.fao.org/faostat/en/\#compare.

\leavevmode\vadjust pre{\hypertarget{ref-faoux2fiiasaGlobalAgroecologicalZones2010}{}}%
FAO/IIASA. (2010). Global {Agro-ecological Zones} ({GAEZ}) ver.3.0
{[}Portal{]}. In \emph{GAEZ v3.0 portal}.
https://www.gaez.iiasa.ac.at/w/ctrl?\_flow=Vwr\&\_view=Welcome\&fieldmain=main\_\&idPS=0\&idAS=0\&idFS=0.

\leavevmode\vadjust pre{\hypertarget{ref-fitzherbertHowWillOil2008}{}}%
Fitzherbert, E., Struebig, M., Morel, A., Danielsen, F., Bruhl, C.,
Donald, P., \& Phalan, B. (2008). How will oil palm expansion affect
biodiversity? \emph{Trends in Ecology \& Evolution}, \emph{23}(10),
538--545. \url{https://doi.org/10.1016/j.tree.2008.06.012}

\leavevmode\vadjust pre{\hypertarget{ref-foleySolutionsCultivatedPlanet2011}{}}%
Foley, J. A., Ramankutty, N., Brauman, K. A., Cassidy, E. S., Gerber, J.
S., Johnston, M., Mueller, N. D., O'Connell, C., Ray, D. K., West, P.
C., Balzer, C., Bennett, E. M., Carpenter, S. R., Hill, J., Monfreda,
C., Polasky, S., Rockström, J., Sheehan, J., Siebert, S., \ldots{} Zaks,
D. P. M. (2011). Solutions for a cultivated planet. \emph{Nature},
\emph{478}(7369), 337--342. \url{https://doi.org/10.1038/nature10452}

\leavevmode\vadjust pre{\hypertarget{ref-gaveauSlowingDeforestationIndonesia2022}{}}%
Gaveau, D. L. A., Locatelli, B., Salim, M. A., Husnayaen, Manurung, T.,
Descals, A., Angelsen, A., Meijaard, E., \& Sheil, D. (2022). Slowing
deforestation in {Indonesia} follows declining oil palm expansion and
lower oil prices. \emph{PLOS ONE}, \emph{17}(3), e0266178.
\url{https://doi.org/10.1371/journal.pone.0266178}

\leavevmode\vadjust pre{\hypertarget{ref-gaveauRiseFallForest2019}{}}%
Gaveau, D. L. A., Locatelli, B., Salim, M. A., Yaen, H., Pacheco, P., \&
Sheil, D. (2019). Rise and fall of forest loss and industrial
plantations in {Borneo} (2000\textendash 2017). \emph{Conservation
Letters}, \emph{12}(3), e12622. \url{https://doi.org/10.1111/conl.12622}

\leavevmode\vadjust pre{\hypertarget{ref-gaveauFourDecadesForest2014}{}}%
Gaveau, D. L. A., Sloan, S., Molidena, E., Yaen, H., Sheil, D., Abram,
N. K., Ancrenaz, M., Nasi, R., Quinones, M., Wielaard, N., \& Meijaard,
E. (2014). Four {Decades} of {Forest Persistence}, {Clearance} and
{Logging} on {Borneo}. \emph{PLoS ONE}, \emph{9}(7), e101654.
\url{https://doi.org/10.1371/journal.pone.0101654}

\leavevmode\vadjust pre{\hypertarget{ref-gilliesRasterioDocumentation2023}{}}%
Gillies, S. (2023). \emph{Rasterio {Documentation}}.

\leavevmode\vadjust pre{\hypertarget{ref-greenpeaceinternationalCertifyingDestruction2013}{}}%
Greenpeace International. (2013). \emph{Certifying destruction}.

\leavevmode\vadjust pre{\hypertarget{ref-groganGlobalGriddedCrop2022}{}}%
Grogan, D., Frolking, S., Wisser, D., Prusevich, A., \& Glidden, S.
(2022). Global gridded crop harvested area, production, yield, and
monthly physical area data circa 2015. \emph{Scientific Data},
\emph{9}(1), 15. \url{https://doi.org/10.1038/s41597-021-01115-2}

\leavevmode\vadjust pre{\hypertarget{ref-haddadHabitatFragmentationIts2015a}{}}%
Haddad, N. M., Brudvig, L. A., Clobert, J., Davies, K. F., Gonzalez, A.,
Holt, R. D., Lovejoy, T. E., Sexton, J. O., Austin, M. P., Collins, C.
D., Cook, W. M., Damschen, E. I., Ewers, R. M., Foster, B. L., Jenkins,
C. N., King, A. J., Laurance, W. F., Levey, D. J., Margules, C. R.,
\ldots{} Townshend, J. R. (2015). Habitat fragmentation and its lasting
impact on {Earth}'s ecosystems. \emph{Science Advances}, \emph{1}(2),
e1500052. \url{https://doi.org/10.1126/sciadv.1500052}

\leavevmode\vadjust pre{\hypertarget{ref-hadiyantoPhytoremediationsPalmOil2013}{}}%
Hadiyanto, H., Christward, M., \& Soetrisnan, D. (2013).
Phytoremediations of {Palm Oil Mill Effluent} ({POME}) by {Using Aquatic
Plants} and {Microalge} for {Biomass Production}. \emph{Journal of
Environmental Science and Technology}, \emph{6}(2), 79--90.
\url{https://doi.org/10.3923/jest.2013.79.90}

\leavevmode\vadjust pre{\hypertarget{ref-hansenHighResolutionGlobalMaps2013}{}}%
Hansen, M. C., Potapov, P. V., Moore, R., Hancher, M., Turubanova, S.
A., Tyukavina, A., Thau, D., Stehman, S. V., Goetz, S. J., Loveland, T.
R., Kommareddy, A., Egorov, A., Chini, L., Justice, C. O., \& Townshend,
J. R. G. (2013). High-{Resolution Global Maps} of 21st-{Century Forest
Cover Change}. \emph{Science}, \emph{342}(6160), 850--853.
\url{https://doi.org/10.1126/science.1244693}

\leavevmode\vadjust pre{\hypertarget{ref-harrisArrayProgrammingNumPy2020}{}}%
Harris, C. R., Millman, K. J., Van Der Walt, S. J., Gommers, R.,
Virtanen, P., Cournapeau, D., Wieser, E., Taylor, J., Berg, S., Smith,
N. J., Kern, R., Picus, M., Hoyer, S., Van Kerkwijk, M. H., Brett, M.,
Haldane, A., Del Río, J. F., Wiebe, M., Peterson, P., \ldots{} Oliphant,
T. E. (2020). Array programming with {NumPy}. \emph{Nature},
\emph{585}(7825), 357--362.
\url{https://doi.org/10.1038/s41586-020-2649-2}

\leavevmode\vadjust pre{\hypertarget{ref-harrisonTropicalForestPeatland2020}{}}%
Harrison, M. E., Ottay, J. B., D'Arcy, L. J., Cheyne, S. M., Anggodo,
Belcher, C., Cole, L., Dohong, A., Ermiasi, Y., Feldpausch, T.,
Gallego-Sala, A., Gunawan, A., Höing, A., Husson, S. J., Kulu, I. P.,
Soebagio, S. M., Mang, S., Mercado, L., Morrogh-Bernard, H. C., \ldots{}
Van Veen, F. J. F. (2020). Tropical forest and peatland conservation in
{Indonesia}: {Challenges} and directions. \emph{People and Nature},
\emph{2}(1), 4--28. \url{https://doi.org/10.1002/pan3.10060}

\leavevmode\vadjust pre{\hypertarget{ref-houghtonEmissionsCarbonDeforestation2013}{}}%
Houghton, R. A. (2013). The emissions of carbon from deforestation and
degradation in the tropics: Past trends and future potential.
\emph{Carbon Management}, \emph{4}(5), 539--546.
\url{https://doi.org/10.4155/cmt.13.41}

\leavevmode\vadjust pre{\hypertarget{ref-ieaGlobalLandTransport2013}{}}%
IEA. (2013). \emph{Global {Land Transport Infrastructure Requirements}}.

\leavevmode\vadjust pre{\hypertarget{ref-iucnPongoPygmaeusAncrenaz2016}{}}%
IUCN. (2016). \emph{Pongo pygmaeus: {Ancrenaz}, {M}., {Gumal}, {M}.,
{Marshall}, {A}.{J}., {Meijaard}, {E}., {Wich} , {S}.{A}. \& {Husson},
{S}.: {The IUCN Red List} of {Threatened Species} 2016:
E.{T17975A123809220}}. {International Union for Conservation of Nature}.
\url{https://doi.org/10.2305/IUCN.UK.2016-1.RLTS.T17975A17966347.en}

\leavevmode\vadjust pre{\hypertarget{ref-jaureguiberryDirectDriversRecent2022}{}}%
Jaureguiberry, P., Titeux, N., Wiemers, M., Bowler, D. E., Coscieme, L.,
Golden, A. S., Guerra, C. A., Jacob, U., Takahashi, Y., Settele, J.,
Díaz, S., Molnár, Z., \& Purvis, A. (2022). The direct drivers of recent
global anthropogenic biodiversity loss. \emph{Science Advances},
\emph{8}(45), eabm9982. \url{https://doi.org/10.1126/sciadv.abm9982}

\leavevmode\vadjust pre{\hypertarget{ref-jenkinsGlobalPatternsTerrestrial2013}{}}%
Jenkins, C. N., Pimm, S. L., \& Joppa, L. N. (2013). Global patterns of
terrestrial vertebrate diversity and conservation. \emph{Proceedings of
the National Academy of Sciences}, \emph{110}(28).
\url{https://doi.org/10.1073/pnas.1302251110}

\leavevmode\vadjust pre{\hypertarget{ref-jeongPerformanceComparisonMesophilic2014}{}}%
Jeong, J.-Y., Son, S.-M., Pyon, J.-H., \& Park, J.-Y. (2014).
Performance comparison between mesophilic and thermophilic anaerobic
reactors for treatment of palm oil mill effluent. \emph{Bioresource
Technology}, \emph{165}, 122--128.
\url{https://doi.org/10.1016/j.biortech.2014.04.007}

\leavevmode\vadjust pre{\hypertarget{ref-joglekarPixelatingCropProduction2019}{}}%
Joglekar, A. K. B., Wood-Sichra, U., \& Pardey, P. G. (2019). Pixelating
crop production: {Consequences} of methodological choices. \emph{PLOS
ONE}, \emph{14}(2), e0212281.
\url{https://doi.org/10.1371/journal.pone.0212281}

\leavevmode\vadjust pre{\hypertarget{ref-kamyabElaeisGuineensis2022}{}}%
Kamyab, H. (Ed.). (2022). \emph{Elaeis guineensis}. {IntechOpen}.
\url{https://doi.org/10.5772/intechopen.92931}

\leavevmode\vadjust pre{\hypertarget{ref-karraGlobalLandUse2021}{}}%
Karra, K., Kontgis, C., Statman-Weil, Z., Mazzariello, J. C., Mathis,
M., \& Brumby, S. P. (2021). Global land use / land cover with
{Sentinel} 2 and deep learning. \emph{2021 {IEEE International
Geoscience} and {Remote Sensing Symposium IGARSS}}, 4704--4707.
\url{https://doi.org/10.1109/IGARSS47720.2021.9553499}

\leavevmode\vadjust pre{\hypertarget{ref-klokantechnologiesgmbhTimbalai1948RSO}{}}%
Klokan Technologies GmbH. (n.d.). \emph{Timbalai 1948 / {RSO Borneo}
({ftSe}) - {EPSG}:29872}. https://epsg.io.

\leavevmode\vadjust pre{\hypertarget{ref-lauranceGlobalStrategyRoad2014}{}}%
Laurance, W. F., Clements, G. R., Sloan, S., O'Connell, C. S., Mueller,
N. D., Goosem, M., Venter, O., Edwards, D. P., Phalan, B., Balmford, A.,
Van Der Ree, R., \& Arrea, I. B. (2014). A global strategy for road
building. \emph{Nature}, \emph{513}(7517), 229--232.
\url{https://doi.org/10.1038/nature13717}

\leavevmode\vadjust pre{\hypertarget{ref-lewanzikArtificialLightPuts2014}{}}%
Lewanzik, D., \& Voigt, C. C. (2014). Artificial light puts ecosystem
services of frugivorous bats at risk. \emph{Journal of Applied Ecology},
\emph{51}(2), 388--394. \url{https://doi.org/10.1111/1365-2664.12206}

\leavevmode\vadjust pre{\hypertarget{ref-lyonsWhyIndonesiaMoving2019}{}}%
Lyons, K. (2019). Why is {Indonesia} moving its capital city?
{Everything} you need to know. \emph{The Guardian}.

\leavevmode\vadjust pre{\hypertarget{ref-mackinnonEcologyKalimantan1997}{}}%
MacKinnon, K., Hatta, G., Halim, H., \& Mangalik, A. (1997). \emph{The
ecology at {Kalimantan}}. {Oxford University Press}.

\leavevmode\vadjust pre{\hypertarget{ref-matricardiLongtermForestDegradation2020}{}}%
Matricardi, E. A. T., Skole, D. L., Costa, O. B., Pedlowski, M. A.,
Samek, J. H., \& Miguel, E. P. (2020). Long-term forest degradation
surpasses deforestation in the {Brazilian Amazon}. \emph{Science},
\emph{369}(6509), 1378--1382.
\url{https://doi.org/10.1126/science.abb3021}

\leavevmode\vadjust pre{\hypertarget{ref-meijaardCoconutOilConservation2020}{}}%
Meijaard, E., Abrams, J. F., Juffe-Bignoli, D., Voigt, M., \& Sheil, D.
(2020). Coconut oil, conservation and the conscientious consumer.
\emph{Current Biology}, \emph{30}(13), R757--R758.
\url{https://doi.org/10.1016/j.cub.2020.05.059}

\leavevmode\vadjust pre{\hypertarget{ref-meijerGlobalPatternsCurrent2018}{}}%
Meijer, J. R., Huijbregts, M. A. J., Schotten, K. C. G. J., \& Schipper,
A. M. (2018). Global patterns of current and future road infrastructure.
\emph{Environmental Research Letters}, \emph{13}(6), 064006.
\url{https://doi.org/10.1088/1748-9326/aabd42}

\leavevmode\vadjust pre{\hypertarget{ref-miller-rushingHowDoesHabitat2019}{}}%
Miller-Rushing, A. J., Primack, R. B., Devictor, V., Corlett, R. T.,
Cumming, G. S., Loyola, R., Maas, B., \& Pejchar, L. (2019). How does
habitat fragmentation affect biodiversity? {A} controversial question at
the core of conservation biology. \emph{Biological Conservation},
\emph{232}, 271--273. \url{https://doi.org/10.1016/j.biocon.2018.12.029}

\leavevmode\vadjust pre{\hypertarget{ref-mobasheriHighlightingRecentTrends2020}{}}%
Mobasheri, A., Mitasova, H., Neteler, M., Singleton, A., Ledoux, H., \&
Brovelli, M. A. (2020). Highlighting recent trends in open source
geospatial science and software. \emph{Transactions in GIS},
\emph{24}(5), 1141--1146. \url{https://doi.org/10.1111/tgis.12703}

\leavevmode\vadjust pre{\hypertarget{ref-mobasheriOpensourceGeospatialTools2020}{}}%
Mobasheri, A., Pirotti, F., \& Agugiaro, G. (2020). Open-source
geospatial tools and technologies for urban and environmental studies.
\emph{Open Geospatial Data, Software and Standards}, \emph{5}(1), 5,
s40965-020-00078-2. \url{https://doi.org/10.1186/s40965-020-00078-2}

\leavevmode\vadjust pre{\hypertarget{ref-mooreAreRangerPatrols2018}{}}%
Moore, J. F., Mulindahabi, F., Masozera, M. K., Nichols, J. D., Hines,
J. E., Turikunkiko, E., \& Oli, M. K. (2018). Are ranger patrols
effective in reducing poaching-related threats within protected areas?
\emph{Journal of Applied Ecology}, \emph{55}(1), 99--107.
\url{https://doi.org/10.1111/1365-2664.12965}

\leavevmode\vadjust pre{\hypertarget{ref-mungiRoleSpeciesRichness2021}{}}%
Mungi, N. A., Qureshi, Q., \& Jhala, Y. V. (2021). Role of species
richness and human impacts in resisting invasive species in tropical
forests. \emph{Journal of Ecology}, \emph{109}(9), 3308--3321.
\url{https://doi.org/10.1111/1365-2745.13751}

\leavevmode\vadjust pre{\hypertarget{ref-murphyOilPalm2020s2021}{}}%
Murphy, D. J., Goggin, K., \& Paterson, R. R. M. (2021). Oil palm in the
2020s and beyond: Challenges and solutions. \emph{CABI Agriculture and
Bioscience}, \emph{2}(1), 39.
\url{https://doi.org/10.1186/s43170-021-00058-3}

\leavevmode\vadjust pre{\hypertarget{ref-myersBiodiversityHotspotsConservation2000}{}}%
Myers, N., Mittermeier, R. A., Mittermeier, C. G., Da Fonseca, G. A. B.,
\& Kent, J. (2000). Biodiversity hotspots for conservation priorities.
\emph{Nature}, \emph{403}(6772), 853--858.
\url{https://doi.org/10.1038/35002501}

\leavevmode\vadjust pre{\hypertarget{ref-nasiForestEcosystemServices2002}{}}%
Nasi, R., Wunder, S., \& Campos A., J. J. (2002). Forest ecosystem
services: Can they pay our way out of deforestation? \emph{CIFOR for the
Global Environmental Facility (GEF)}.

\leavevmode\vadjust pre{\hypertarget{ref-oecdOECDFAOAgriculturalOutlook2023}{}}%
OECD. (2023). \emph{{OECD-FAO Agricultural Outlook} 2023-2032}. {Food
and Agriculture Organization of the United Nations}.
\url{https://doi.org/10.1787/08801ab7-en}

\leavevmode\vadjust pre{\hypertarget{ref-omettoContributionWorkingGroup2022}{}}%
Ometto, J. P., Kalaba, K., Anshari, G. Z., Chacón, N., Farrell, A.,
Halim, S. A., Neufeldt, H., \& Sukumar, R. (2022). \emph{Contribution of
{Working Group II} to the {Sixth Assessment Report} of the
{Intergovernmental Panel} on {Climate Change}}. {Cambridge University
Press}. \url{https://doi.org/10.1017/9781009325844.024}

\leavevmode\vadjust pre{\hypertarget{ref-potapovGlobal20002020Land2022}{}}%
Potapov, P., Hansen, M. C., Pickens, A., Hernandez-Serna, A., Tyukavina,
A., Turubanova, S., Zalles, V., Li, X., Khan, A., Stolle, F., Harris,
N., Song, X.-P., Baggett, A., Kommareddy, I., \& Kommareddy, A. (2022).
The {Global} 2000-2020 {Land Cover} and {Land Use Change Dataset Derived
From} the {Landsat Archive}: {First Results}. \emph{Frontiers in Remote
Sensing}, \emph{3}, 856903.
\url{https://doi.org/10.3389/frsen.2022.856903}

\leavevmode\vadjust pre{\hypertarget{ref-potapovGlobalMapsCropland2021}{}}%
Potapov, P., Turubanova, S., Hansen, M. C., Tyukavina, A., Zalles, V.,
Khan, A., Song, X.-P., Pickens, A., Shen, Q., \& Cortez, J. (2021).
Global maps of cropland extent and change show accelerated cropland
expansion in the twenty-first century. \emph{Nature Food}, \emph{3}(1),
19--28. \url{https://doi.org/10.1038/s43016-021-00429-z}

\leavevmode\vadjust pre{\hypertarget{ref-puttkerIndirectEffectsHabitat2020}{}}%
Püttker, T., Crouzeilles, R., Almeida-Gomes, M., Schmoeller, M.,
Maurenza, D., Alves-Pinto, H., Pardini, R., Vieira, M. V., Banks-Leite,
C., Fonseca, C. R., Metzger, J. P., Accacio, G. M., Alexandrino, E. R.,
Barros, C. S., Bogoni, J. A., Boscolo, D., Brancalion, P. H. S., Bueno,
A. A., Cambui, E. C. B., \ldots{} Prevedello, J. A. (2020). Indirect
effects of habitat loss via habitat fragmentation: {A} cross-taxa
analysis of forest-dependent species. \emph{Biological Conservation},
\emph{241}, 108368. \url{https://doi.org/10.1016/j.biocon.2019.108368}

\leavevmode\vadjust pre{\hypertarget{ref-ritchiePalmOil2021}{}}%
Ritchie, H. (2021). Palm {Oil}. In \emph{Our World in Data}.
https://ourworldindata.org/palm-oil.

\leavevmode\vadjust pre{\hypertarget{ref-rochmyaningsihClaimThatCoconut2020}{}}%
Rochmyaningsih, D. (2020). Claim that coconut oil is worse for
biodiversity than palm oil sparks furious debate. \emph{Science}.
\url{https://doi.org/10.1126/science.abd8820}

\leavevmode\vadjust pre{\hypertarget{ref-rspoRSPOPrinciplesCriteria2018}{}}%
RSPO. (2018). \emph{{RSPO Principles} \& {Criteria} for the {Production}
of {Sustainable Palm Oi}}.

\leavevmode\vadjust pre{\hypertarget{ref-rspoWhoWeAre2023}{}}%
RSPO. (2023). Who we are. In \emph{Roundtable on Sustainable Palm Oil
(RSPO)}. https://rspo.org/who-we-are/.

\leavevmode\vadjust pre{\hypertarget{ref-sarafanovMachineLearningApproach2020}{}}%
Sarafanov, M., Kazakov, E., Nikitin, N. O., \& Kalyuzhnaya, A. V.
(2020). A {Machine Learning Approach} for {Remote Sensing Data
Gap-Filling} with {Open-Source Implementation}: {An Example Regarding
Land Surface Temperature}, {Surface Albedo} and {NDVI}. \emph{Remote
Sensing}, \emph{12}(23), 3865. \url{https://doi.org/10.3390/rs12233865}

\leavevmode\vadjust pre{\hypertarget{ref-sheilImpactsOpportunitiesOil2009}{}}%
Sheil, D. (Ed.). (2009). \emph{The impacts and opportunities of oil palm
in {Southeast Asia}: What do we know and what do we need to know?}
{CIFOR}.

\leavevmode\vadjust pre{\hypertarget{ref-simedarbyplantationSimeDarbyPlantation2020}{}}%
Sime Darby Plantation. (2020). \emph{Sime {Darby Plantation Publishes}
its {Oil Palm Genome} to {Support The Company}'s {Ambition} for a
{Deforestation-Free Industry}}.
https://simedarbyplantation.com/sime-darby-plantation-publishes-its-oil-palm-genome-to-support-the-companys-ambition-for-a-deforestation-free-industry/.

\leavevmode\vadjust pre{\hypertarget{ref-sloanHiddenChallengesConservation2019}{}}%
Sloan, S., Campbell, M. J., Alamgir, M., Engert, J., Ishida, F. Y.,
Senn, N., Huther, J., \& Laurance, W. F. (2019). Hidden challenges for
conservation and development along the {Trans-Papuan} economic corridor.
\emph{Environmental Science \& Policy}, \emph{92}, 98--106.
\url{https://doi.org/10.1016/j.envsci.2018.11.011}

\leavevmode\vadjust pre{\hypertarget{ref-sweeneyStreamsideForestBuffer2014}{}}%
Sweeney, B. W., \& Newbold, J. D. (2014). Streamside {Forest Buffer
Width Needed} to {Protect Stream Water Quality}, {Habitat}, and
{Organisms}: {A Literature Review}. \emph{JAWRA Journal of the American
Water Resources Association}, \emph{50}(3), 560--584.
\url{https://doi.org/10.1111/jawr.12203}

\leavevmode\vadjust pre{\hypertarget{ref-turubanovaOngoingPrimaryForest2018}{}}%
Turubanova, S., Potapov, P. V., Tyukavina, A., \& Hansen, M. C. (2018).
Ongoing primary forest loss in {Brazil}, {Democratic Republic} of the
{Congo}, and {Indonesia}. \emph{Environmental Research Letters},
\emph{13}(7), 074028. \url{https://doi.org/10.1088/1748-9326/aacd1c}

\leavevmode\vadjust pre{\hypertarget{ref-tyukavinaGlobalTrendsForest2022}{}}%
Tyukavina, A., Potapov, P., Hansen, M. C., Pickens, A. H., Stehman, S.
V., Turubanova, S., Parker, D., Zalles, V., Lima, A., Kommareddy, I.,
Song, X.-P., Wang, L., \& Harris, N. (2022). Global {Trends} of {Forest
Loss Due} to {Fire From} 2001 to 2019. \emph{Frontiers in Remote
Sensing}, \emph{3}, 825190.
\url{https://doi.org/10.3389/frsen.2022.825190}

\leavevmode\vadjust pre{\hypertarget{ref-unep-wcmcProtectedAreaProfile2023}{}}%
UNEP-WCMC. (2023). \emph{Protected {Area Profile} for {Asia} \&
{Pacific} from the {World Database} on {Protected Areas}}.
https://www.protectedplanet.net/.

\leavevmode\vadjust pre{\hypertarget{ref-voigtDeforestationProjectionsImply2022}{}}%
Voigt, M., Kühl, H. S., Ancrenaz, M., Gaveau, D., Meijaard, E., Santika,
T., Sherman, J., Wich, S. A., Wolf, F., Struebig, M. J., Pereira, H. M.,
\& Rosa, I. M. D. (2022). Deforestation projections imply range-wide
population decline for critically endangered {Bornean} orangutan.
\emph{Perspectives in Ecology and Conservation}, \emph{20}(3), 240--248.
\url{https://doi.org/10.1016/j.pecon.2022.06.001}

\leavevmode\vadjust pre{\hypertarget{ref-waringForestsDecarbonizationRoles2020}{}}%
Waring, B., Neumann, M., Prentice, I. C., Adams, M., Smith, P., \&
Siegert, M. (2020). Forests and {Decarbonization} \textendash{} {Roles}
of {Natural} and {Planted Forests}. \emph{Frontiers in Forests and
Global Change}, \emph{3}, 58.
\url{https://doi.org/10.3389/ffgc.2020.00058}

\leavevmode\vadjust pre{\hypertarget{ref-wassmannPalmOilRoundtable2023}{}}%
Wassmann, B., Siegrist, M., \& Hartmann, C. (2023). Palm oil and the
{Roundtable} of {Sustainable Palm Oil} ({RSPO}) label: {Are Swiss}
consumers aware and concerned? \emph{Food Quality and Preference},
\emph{103}, 104686. \url{https://doi.org/10.1016/j.foodqual.2022.104686}

\leavevmode\vadjust pre{\hypertarget{ref-wood-sichraSpatialProductionAllocation2016}{}}%
Wood-Sichra, U., Joglekar, A. B., \& You, L. (2016). \emph{Spatial
{Production Allocation Model} ({SPAM}) 2005: {Technical Documentation}}.

\leavevmode\vadjust pre{\hypertarget{ref-worldbankAveragePricesPalm2023}{}}%
World Bank. (2023). \emph{Average prices for palm oil worldwide from
2014 to 2024 (in nominal {U}.{S}. Dollars per mt) {[}{Graph}{]}}.
{Statista}.

\leavevmode\vadjust pre{\hypertarget{ref-youMappingGlobalCropping2022}{}}%
You, L., \& Sun, Z. (2022). Mapping global cropping system:
{Challenges}, opportunities, and future perspectives. \emph{Crop and
Environment}, \emph{1}(1), 68--73.
\url{https://doi.org/10.1016/j.crope.2022.03.006}

\leavevmode\vadjust pre{\hypertarget{ref-youGeneratingPlausibleCrop2009}{}}%
You, L., Wood, S., \& Wood-Sichra, U. (2009). Generating plausible crop
distribution maps for {Sub-Saharan Africa} using a spatially
disaggregated data fusion and optimization approach. \emph{Agricultural
Systems}, \emph{99}(2-3), 126--140.
\url{https://doi.org/10.1016/j.agsy.2008.11.003}

\leavevmode\vadjust pre{\hypertarget{ref-yuCultivatedPlanet20102020}{}}%
Yu, Q., You, L., Wood-Sichra, U., Ru, Y., Joglekar, A. K. B., Fritz, S.,
Xiong, W., Lu, M., Wu, W., \& Yang, P. (2020). A cultivated planet in
2010 \textendash{} {Part} 2: {The} global gridded
agricultural-production maps. \emph{Earth System Science Data},
\emph{12}(4), 3545--3572.
\url{https://doi.org/10.5194/essd-12-3545-2020}

\leavevmode\vadjust pre{\hypertarget{ref-zafirahSustainableEcosystemServices2017}{}}%
Zafirah, N., Nurin, N. A., Samsurijan, M. S., Zuknik, M. H., Rafatullah,
M., \& Syakir, M. I. (2017). Sustainable {Ecosystem Services Framework}
for {Tropical Catchment Management}: {A Review}. \emph{Sustainability},
\emph{9}(4), 546. \url{https://doi.org/10.3390/su9040546}

\leavevmode\vadjust pre{\hypertarget{ref-zanagaESAWorldCover102021}{}}%
Zanaga, D., Van De Kerchove, R., De Keersmaecker, W., Souverijns, N.,
Brockmann, C., Quast, R., Wevers, J., Grosu, A., Paccini, A., Vergnaud,
S., Cartus, O., Santoro, M., Fritz, S., Georgieva, I., Lesiv, M.,
Carter, S., Herold, M., Li, L., Tsendbazar, N.-E., \ldots{} Arino, O.
(2021). \emph{{ESA WorldCover} 10 m 2020 V100}. {Zenodo}.
\url{https://doi.org/10.5281/ZENODO.5571936}

\leavevmode\vadjust pre{\hypertarget{ref-zhuBenefitsFreeOpen2019}{}}%
Zhu, Z., Wulder, M. A., Roy, D. P., Woodcock, C. E., Hansen, M. C.,
Radeloff, V. C., Healey, S. P., Schaaf, C., Hostert, P., Strobl, P.,
Pekel, J.-F., Lymburner, L., Pahlevan, N., \& Scambos, T. A. (2019).
Benefits of the free and open {Landsat} data policy. \emph{Remote
Sensing of Environment}, \emph{224}, 382--385.
\url{https://doi.org/10.1016/j.rse.2019.02.016}

\end{CSLReferences}

\listoffigures

\listoftables

\bookmarksetup{startatroot}

\hypertarget{annex}{%
\chapter*{Annex}\label{annex}}

\markboth{Annex}{Annex}

\pagenumbering{gobble}

\hypertarget{detailed-research-questions}{%
\section*{\texorpdfstring{\textsc{I} Detailed research
questions}{ Detailed research questions}}\label{detailed-research-questions}}

\markright{\textsc{I} Detailed research questions}

\textbf{Forest Information 2000} How much forest was there in the year
2000? How much primary forest was there?

\textbf{1 General Forest loss}

1.1 How much forest area was lost yearly and in total?

1.2 How much forest area was lost due to forest fires yearly and in
total?

1.3 How much forest gain (area) occured after forest fires (2001 -
2012)?

1.4 How much forest was lost in protected areas yearly?

1.5 How much forest was lost in protected areas excluding forest fires?

\textbf{2 Primary Forest loss}

2.1 How much primary forest area was lost yearly and in total?

2.2 How much primary forest area was lost due to forest fires yearly and
in total?

2.3 How much primary forest was lost in protected areas yearly?

2.4 How much forest was lost in protected areas excluding forest fires?

\textbf{3 Oil Palm related}

3.1 How much new oil palm plantations occurred yearly (2000 - 2017)?

3.2 How much new oil palm occurred on previously deforested areas? (2001
- 2017)

3.3 How much new oil palm plantation area occured yearly on areas
previously deforested by forest fires?

3.4 How much new oil palm plantation area occured in protected ares?

3.5 How much new oil palm plantation areo occured on non-forest area?
(compared to year 2000 forest cover)

3.6 How much new oil palm plantations occurred yearly on primary forest
(2000 - 2017)?

3.7 How much new oil palm plantation area occured on previos cropland
(and other way around)?

3.8 How much forest area was ganied on previous oil palm plantation area
yearly (2000 - 2012)?

3.9 How much area was used for other crops prior to oil palm plantation,
and which?

3.10 How much area was used for oil palm plantation prior to other
crops, and which?

\textbf{5 Build up areas} 4.1 How Much new build up area was created
from 2000 to 2020?

4.2a How much forest loss areas occured on new build up area (yearly)?

4.2b How much primiary forest loss areas occured on new build up area
(yearly)?

4.3 How much new build up area occured in non-forest covered area (2020
compared to 2000)?

4.4 How much new build up area occured in forest fire area (2020
compared to 2000)?

4.5 How much new oil palm plantation area occured within 0.1, 0.2, 0.5,
1, and 2 km of year 2000, new\_build up and total build up areas?

4.6 How much forest fire area occured within 0.1, 0.2, 0.5, 1, and 2 km
of year 2000, new\_build up and total build up areas?

4.7 How much deforestation (excluding forest fires) area occured within
0.1, 0.2, 0.5, 1, and 2 km of year 2000, new\_build up and total build
up areas?

4.8 How much forest area was lost to cropland areas within 1, 2, 5, 10,
and 20 km of newly buld up areas?

4.9 How much new build up areas was created on primary forest loss areas
(2020 compared to 2000)?

\textbf{6 RSPO}

5.1 How much deforestation occurs within RSPO certified concessions?

\newpage

\hypertarget{overlap-of-75-closed-canopy-and-primary-forest}{%
\section*{\texorpdfstring{\textsc{II} Overlap of 75\% closed canopy and
primary
forest}{ Overlap of 75\% closed canopy and primary forest}}\label{overlap-of-75-closed-canopy-and-primary-forest}}

\markright{\textsc{II} Overlap of 75\% closed canopy and primary forest}

\color{white}

{[}\includegraphics[width=1\textwidth,height=\textheight]{text/../code/results/maps/fcover2000_75_and_primary_forest.png}{]}
\normalcolor

Representative map extract showing that \textgreater75\% closed canopy
of 2000 mostly covers the primary forest of 2001. \newpage

\hypertarget{dense-vegetation-excluding-primary-forests}{%
\section*{\texorpdfstring{\textsc{III} Dense vegetation excluding
primary
forests}{ Dense vegetation excluding primary forests}}\label{dense-vegetation-excluding-primary-forests}}

\markright{\textsc{III} Dense vegetation excluding primary forests}

\color{white}

{[}\includegraphics[width=1\textwidth,height=\textheight]{text/../code/results/maps/validation_built_up_deforestation.png}{]}
\normalcolor

In the dense vegetation outside of primary forests, lage areas are
occupied by oil palm. Oil palm data is limited to years 2000 - 2003 as
growing oil palms could alread have a closed canopy of \textgreater75\%
in the year 2000. \newpage

\hypertarget{validation-of-built-up-areas-in-2000-and-deforestation}{%
\section*{\texorpdfstring{\textsc{IV} Validation of built up areas in
2000 and
deforestation}{ Validation of built up areas in 2000 and deforestation}}\label{validation-of-built-up-areas-in-2000-and-deforestation}}

\markright{\textsc{IV} Validation of built up areas in 2000 and
deforestation}

\color{white}

{[}\includegraphics[width=1\textwidth,height=\textheight]{text/../code/results/maps/validation_built_up_deforestation.png}{]}
\normalcolor

Example of deforestation on year 2000 built up area, where, logically,
now deforestation should be. Here it is visible, that these areas are
mainly located at the edge, where maintenance led to the removal of
closeby trees, which were thus classifiead as forest loss. \newpage

\hypertarget{secondary-forest-and-proximity-to-rivers}{%
\section*{\texorpdfstring{\textsc{V} Secondary Forest and proximity to
rivers}{ Secondary Forest and proximity to rivers}}\label{secondary-forest-and-proximity-to-rivers}}

\markright{\textsc{V} Secondary Forest and proximity to rivers}

\color{white}

{[}\includegraphics[width=1\textwidth,height=\textheight]{text/../code/results/maps/reason_secondary_rivers.png}{]}
\normalcolor

A repeating pattern, with a lot of secondary forest close to rivers,
especially with built-up areas nearby. \newpage

\hypertarget{secondary-forest-and-proximity-to-rivers-1}{%
\section*{\texorpdfstring{\textsc{V} Secondary Forest and proximity to
rivers}{ Secondary Forest and proximity to rivers}}\label{secondary-forest-and-proximity-to-rivers-1}}

\markright{\textsc{V} Secondary Forest and proximity to rivers}

\color{white}

{[}\includegraphics[width=1\textwidth,height=\textheight]{text/../code/results/maps/deforestation_protected_areas_other.png}{]}
\normalcolor

An exceptional case of a protected PA where much deforestation takes
place. A prominent feature is a new road that cuts through the PA.
\newpage

\hypertarget{sec-annex_x}{%
\section*{\texorpdfstring{\textsc{X} Plagiarism
declaration}{ Plagiarism declaration}}\label{sec-annex_x}}

\markright{\textsc{X} Plagiarism declaration}

\color{white}

{[}\includegraphics[width=1\textwidth,height=\textheight]{text/annex_files/plagiarism_declaration.png}{]}
\normalcolor \newpage



\end{document}
